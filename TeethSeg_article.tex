\documentclass[10pt]{article}
\usepackage[utf8]{inputenc}
\usepackage{geometry} % to change the page dimensions
\geometry{a4paper}
\usepackage{authblk}

%% Packages
\usepackage{graphicx, subcaption}
\usepackage{amsmath,amssymb,amsthm,amsfonts}
\usepackage{bm}
\usepackage{algorithm, algorithmic}
\usepackage{multirow}
\usepackage{bbm}
\usepackage{hyperref,url}
\usepackage{kotex}
\usepackage[labelformat=simple]{subcaption}

\usepackage{tikz}
\usetikzlibrary{shapes.geometric, arrows, positioning, calc}
\usetikzlibrary{backgrounds}
\usetikzlibrary{arrows.meta}

\tikzstyle{title} = [rectangle, minimum width=3cm, minimum height=1cm, text centered, draw=black, text width=3cm, fill=white!100]
\tikzstyle{startstop} = [rectangle, rounded corners, minimum width=3cm, minimum height=1cm,text centered, draw=black, fill=red!30]
\tikzstyle{io} = [trapezium, trapezium left angle=70, trapezium right angle=110, minimum width=3cm, minimum height=1cm, text width=1.3cm, text centered, draw=black, fill=blue!30]
\tikzstyle{process} = [rectangle, minimum width=3cm, text width=3cm, minimum height=1cm, text centered, draw=black, fill=orange!30]
\tikzstyle{decision} = [diamond, minimum width=3cm, minimum height=1cm, text centered, draw=black, fill=green!30]
\tikzstyle{arrow} = [thick,->,>=stealth]

\captionsetup[figure]{labelsep=period}
% \renewcommand{\thefigure}{\Roman{figure}}
% \captionsetup[subfigure]{labelformat=parens} % default is 'parens'
% \renewcommand{\thesubfigure}{\thefigure.\alph{subfigure}.}
\renewcommand\thesubfigure{(\alph{subfigure})}

%% Theorems
\theoremstyle{plain}
\newtheorem{theorem}{Theorem}[section]
\newtheorem{proposition}[theorem]{Proposition}
\newtheorem{lemma}[theorem]{Lemma}

\theoremstyle{remark}
\newtheorem{remark}[theorem]{Remark}
\newtheorem{example}[theorem]{Example}

\theoremstyle{definition}
\newtheorem{definition}[theorem]{Definition}

%% Equations
\numberwithin{equation}{section}

%% Algorithms
\renewcommand\algorithmicdo{}
\renewcommand\algorithmicthen{}
\renewcommand\algorithmicendfor{\textbf{end}}

%% Macros
\def\chfun{\mathbbm{1}}
\def\calc{\mathcal{C}}
\def\endske{E\left(\mathcal{S}^8\right)}
\def\endomc{E\left(\Omega_c\right)}


\def\gphi{\nabla\phi}
\def\ngphi{\left|\nabla\phi\right|}
\def\ngphii{\left|\nabla\phi_i\right|}

\def\fci{F_{c,\,i}}
\def\fcj{F_{c,\,j}}

\def\fsi{F_{s,\,i}}

\def\cm{\, ,}
\def\pd{\, .}

\def\omi{\Omega_i}
\def\oma{\Omega_a}
\def\ome{\Omega_E}
\def\oms{\Omega_S}
\def\omc{\overline{\Omega}_P}
\def\omcc{\Omega_P}
\def\omt{\Omega_T}
\def\omn{\Omega_N}

\def\ER{\mathcal{R}}
\def\PER{\mathcal{P}}
\def\ERC{\Omega \setminus \ER}

\def\opp{\Omega_{\phi_i}^+}
\def\opn{\Omega_{\phi_i}^-}
% \def\opo{\Omega_{\phi}^0}
\def\opo{C_{\phi_i}}

\def\sdf{sdf}

\DeclareMathOperator*{\argmax}{arg\,max}
\DeclareMathOperator*{\argmin}{arg\,min}
% \def\tOmega{\tilde{\Omega}}
% \def\tGamma{\tilde{\Gamma}}
% \def\tV{\tilde{V}}
% \def\W{\mathbb{W}}
% \def\tu{\tilde{u}}
% \def\bu{\bar{u}}
% \def\hu{\hat{u}}
% \def\bl{\bar{\lambda}}
% \def\p{\mathbf{p}}
% \def\P{\mathbf{P}}
% \def\intO{\int_{\Omega}}
% \def\intOs{\int_{\Omega_s}}
% \def\m{\mathbf{m}}
% \def\n{\mathbf{n}}
% \def\blambda{\bm{\lambda}}

% \def\tE{\tilde{E}}
% \def\N{\mathcal{N}}

% \def\div{\mathrm{div}}
% \def\proj{\mathrm{proj}}
% \def\prox{\mathrm{prox}}
% \def\ran{\mathrm{ran}\,}
% \def\ed{\mathrm{ed}}
% \def\supp{\mathrm{supp}\,}
% \def\TOL{\mathrm{TOL}}
% \DeclareMathOperator*{\argmin}{\arg\min}

% Text Color and Strike
\usepackage[normalem]{ulem}
\usepackage{color}
\newcommand{\red}[1]{{\color{red}{#1}}}
\newcommand{\blue}[1]{{\color{blue}{#1}}}

\title{ Individual Tooth Segmentation in Human Teeth Images Using Pseudo Edge-Regions Obtained by Deep Neural Networks }
\author{Seongeun Kim and Chang-Ock Lee}
\affil{Department of Mathematical Sciences, KAIST, Daejeon 34141, Korea}
\date{ }

\begin{document}
\maketitle

\begin{abstract}
In human teeth images taken outside the oral cavity with a general digital camera, it is difficult to segment individual tooth due to common obstacles such as weak edges, intensity inhomogeneities and strong light reflections. In this work, we propose a method for segmenting individual tooth in human teeth images. The key of the method is to obtain pseudo edge-regions using deep neural networks. After additional steps to acquire initial contours for each tooth region, we apply an active contour model to segment the individual tooth. We also present a strategy using existing model-based methods for labeling the data required for neural network training.
\end{abstract}

{\small \textbf{Key words}
Tooth segmentation, Neural network, Geometric attraction--driven flow, Edge-region, Light reflection}

{\small \textbf{AMS subject classifications}
94A08, 68T07}

%% Main text starts ---------------------------------------------------------------------------------------------------
% Section: Introduction
\section{Introduction}
\label{Sec:Introduction}

Image segmentation is one of major topics in the field of computer vision. The main goal of image segmentation is to segment objects in an image according to a specific purpose and many outstanding methodologies have been developed. The active contour model is one of the popular methods for the image segmentation and boundary detection. It makes contours move to the boundary of the nearby object~\cite{Kass:1988:Snakes}; gradient vector flows (GVF)~\cite{Xu:1997:GVF}, geodesic active contours (GAC)~\cite{Caselles:1997:GAC} are well-known models with different moving forces. In \cite{Chan:2001:chanvese}, a region-based active contour method was proposed unlike the previous methods using edge maps. Details on these models are elaborated in Section~\ref{Subsec:active_contour}.

Segmentation of individual tooth in human teeth images has been mainly performed on computed tomography~(CT) or X-ray images~\cite{Yuan:2020:teethParanomic,Naumovich:2015:teethCT3D,Said:2006:teethxray,Shah:2006:teethautoseg}. Teeth are easily segmented in these CT or X-ray images since the intensity of the teeth region is much stronger than the gums or tissues. However, segmenting individual tooth in a teeth image taken outside the oral cavity with a general digital camera is a more difficult problem. In the teeth images, not only weak edges and light reflections appear, but also colors and illuminance are inhomogeneous due to plaque and oral structure, which are major obstacles to segmentation of individual tooth. Several attempts have been made to segment individual tooth in teeth images. In~\cite{Na:2014LteethMorph}, the watershed algorithm~\cite{Vincent:1991:watershed} with color information was proposed. Recently, segmentations using supervised learning methods are mainly performed. In~\cite{WU:2016}, a supervised learning method BEL~\cite{Dollar:2006:BEL} was used to obtain the individual tooth boundaries, and in~\cite{Kim:2020,Pham:2020,Zhu:2020:teethMaskrcnn}, deep learning methods based on Mask R-CNN~\cite{He:2018:MRCNN} were proposed. Although the model-based methods are reliable, it produces poor results when applied to images obtained in an unrefined environment. On the other hand, it is well known that supervised learning methods can give excellent results for a variety of images, but require sufficient amount of training data for stable performance, and that the labeling process of training data takes a lot of time and cost.

The main applications of individual tooth segmentation are reconstruction of 3D tooth models, dental identification of deceased individuals performed in forensic dentistry~\cite{Pretty:2001:forensic}, and dental healthcare. The 3D teeth model reconstruction was developed based on individual tooth segmentation in CT images~\cite{Yuan:2020:teethParanomic,Naumovich:2015:teethCT3D}. However, in a recent study~\cite{WU:2016}, a 3D teeth model reconstruction method using individual tooth boundaries in a single teeth image was proposed. In addition, identification of decease is one of the most important field in forensic dentistry to the point where the FBI has created a dedicated organization to operate automatic tooth recognition system~\cite{Shah:2006:teethautoseg}. This is established by comparison of antemortem and postmortem dental records, and an individual tooth segmentation is necessary in this process~\cite{Pretty:2001:forensic}.

In this paper, we segment individual tooth regions using the deep neural network and active contour model. First, we obtain the pseudo edge-region using a deep neural network. When preparing data for training the networks, manually select regions on the tooth boundary among edge-regions acquired by geometric attraction-driven flow (GADF). This greatly reduces the time and cost of the data labeling process compared to performing the entire process manually. After that, active contours model using multiple contours are applied to acquire initial contours, and then segment regions from the pseudo edge-region. Finally, by identifying tooth and non-tooth regions, only the tooth region is obtained as a result.

The remainder of this paper is organized as follows. In Section~\ref{Sec:er_teeth}, we will discuss geometric attraction-driven flow (GADF), edge-regions, and problems of edge-regions obtained from teeth images. In Section~\ref{Sec:PM}, the proposed algorithm is described, and the experimental results are presented in Section~\ref{Sec:result}. We will conclude this paper in Section~\ref{Sec:Conclusion}. 

% Sec:Background
\section{GADF, edge-regions and segmentations}
\label{Sec:er_teeth}

In this section, we briefly review the GADF, edge-region in~\cite{Hahn:2006,Hahn:2010:GADF}, and active contour models, which play key roles in segmenting individual tooth. Then we will show that the edge-regions obtained by GADF are incomplete due to the problematic features of the teeth image.

% Subsec:GADF
\subsection{GADF}
\label{Subsec:GADF}
Let $I\colon\Omega\subset\mathbf{R}^2\rightarrow[0,\,1]$ be a given smooth gray image. The image intensity changes rapidly on the boundary of objects in the image. Thus, a point $x\in\Omega$ is defined as an edge point if
\begin{align*}
    u_x''(0)=0\cm %label{Def:edge}
\end{align*}
where
\begin{align*}
    u_x(s) = I\left( x + s\frac{\nabla{I(x)}}{|\nabla{I(x)}|} \right)\pd
\end{align*}
In \cite{Hahn:2010:GADF}, the GADF is defined as
\begin{align}
    F_a(x) = \mbox{sgn}(\ell(x))\frac{\nabla I(x)}{\left|\nabla{I(x)}\right|},\quad \forall{x}\in\Omega\cm  \label{Def:gadf}
\end{align}
where $\mbox{sgn}$ is the sign function and 
\begin{align*}
    \ell(x) &= \int^{\epsilon}_{0} {u'_x(s)}\,ds - \int^{0}_{-\epsilon} {u'_x(s)}\,ds\quad\mbox{for a small}~\epsilon > 0\pd
\end{align*}
By the definition, $F_a(x)$ points along the line $x + s\frac{\nabla{I(x)}}{|\nabla{I(x)}|}$ towards the edge point. The GADF can be naturally extended to color images; see~\cite{Hahn:2006,Hahn:2010:GADF}, but in this paper, we consider the GADF only for grayscale images which are obtained by averaging channels of given color images.

% Subsection: Edge-regions
\subsection{Edge-regions}
\label{Subsec:edge-regions}

The edge-region is roughly defined as the curve of thin regions containing most of object boundaries, regardless of the strength of edges. In~\cite{Hahn:2010:GADF}, $\ome$ is defined as a region where $F_a$ faces each other, i.e. 
\begin{align*}
    \ome = \left\{ x\in\Omega \mid F_a(x^*) \cdot F_a(x) < 0 ~\mbox{and}~ x^*=x+F_a(x) \right\}\pd %\label{Def:pre_er}
\end{align*}
Since $F_a$ is defined using normalized image gradient, it is independent of the strength of edges so that $\ome$ satisfies the rough definition of the edge-region. But in practice, $\ome$ is just a candidate for the edge-region, because it contains many regions that are not intersected with the boundaries of the object. Hence it is necessary to remove them as many as possible.

In Figure~\ref{Fig:edge_regions}, there are three teeth images: a synthetic image and two real teeth images. For the synthetic image, $\ome$ is well formed only at the tooth boundary, but in the teeth images, there are a lot of regions that do not intersect tooth boundaries. Among the connected components of $\ome$, those who are having small size or formed upon flat image intensities are assumed not to contain tooth boundaries and can be easily removed~\cite{Hahn:2006}. In the rightmost column, refined and dilated $\ome$ using the similar idea to~\cite{Hahn:2006} are presented. Despite these efforts, there are still many unnecessary regions in the refined edge-regions; see the bottom right of~\ref{Fig:edge_regions}. We will talk about this region in the next two sections. %The regions with non-closed shape can be treated applying active contour model~\ref{Subsec:active_contour}, and about the regions with closed shape, the main obstacles of individual tooth segmentation, it will be discussed in Section~\ref{Subsec:light_reflection}.

% Figure: Weak edges in teeth image
\begin{figure}
    \newcommand*{\wid}{3.4}%
    \newcommand*{\twid}{.27}%
    \centering
    \begin{subfigure}{ \twid \textwidth}
        \centering
        \includegraphics[height=\wid cm]{ ./Figures/edge_region/stt.png}
        \includegraphics[height=\wid cm]{ ./Figures/edge_region/nolight.png}
        \includegraphics[height=\wid cm]{ ./Figures/edge_region/light.png}
        \caption{\footnotesize Teeth images}
        %\caption{A synthetic image}
    \end{subfigure} 
    \begin{subfigure}{ \twid \textwidth}
        \centering
        \includegraphics[height=\wid cm]{ ./Figures/edge_region/stt_er.png}
        \includegraphics[height=\wid cm]{ ./Figures/edge_region/nolight_er.png}
        \includegraphics[height=\wid cm]{ ./Figures/edge_region/light_er.png}
        \caption{\footnotesize Edge-region candidate $\ome$}
    \end{subfigure}
    \begin{subfigure}{ \twid \textwidth}
        \centering
        \includegraphics[height=\wid cm]{ ./Figures/edge_region/st_Er_l.png}
        \includegraphics[height=\wid cm]{ ./Figures/edge_region/nolight_Er_l.png}
        \includegraphics[height=\wid cm]{ ./Figures/edge_region/light_Er_l.png}
        \caption{\footnotesize Edge-region $\oma$}
        \label{Fig:edge_regions_4}
    \end{subfigure}

    \caption{ (a) A synthetic image, a teeth image without light reflection, and a teeth image with light reflection~\cite{T001_21} are shown from top to bottom. (b) Edge-region candidates $\ome$ of the left images. (c) Edge-region $\oma$ is refined using component size and image intensity. }
    \label{Fig:edge_regions}
\end{figure}
% \begin{figure}
%     \newcommand*{\wid}{3.0}%
%     \newcommand*{\twid}{.23}%
%     \centering
%     \begin{subfigure}{ \twid \textwidth}
%         \centering
%         \includegraphics[height=\wid cm]{ ./Figures/edge_region/stt.png}
%         \includegraphics[height=\wid cm]{ ./Figures/edge_region/stt_er.png}
%         \includegraphics[height=\wid cm]{ ./Figures/edge_region/stt_Er_l.png}
%         \caption{}
%         %\caption{A synthetic image}
%     \end{subfigure} 
%     \begin{subfigure}{ \twid \textwidth}
%         \centering
%         \includegraphics[height=\wid cm]{ ./Figures/edge_region/nolight.png}
%         \includegraphics[height=\wid cm]{ ./Figures/edge_region/nolight_er.png}
%         \includegraphics[height=\wid cm]{ ./Figures/edge_region/nolight_Er_l.png}
%         \caption{}
%     \end{subfigure}
%     \begin{subfigure}{ \twid \textwidth}
%         \centering
%         \includegraphics[height=\wid cm]{ ./Figures/edge_region/light.png}
%         \includegraphics[height=\wid cm]{ ./Figures/edge_region/light_er.png}
%         \includegraphics[height=\wid cm]{ ./Figures/edge_region/light_Er_l.png}
%         \caption{}
%         \label{Fig:edge_regions_4}
%     \end{subfigure}

%     \caption{ First row: (a) A synthetic image, (b) teeth image without light reflections and (c) teeth image with light reflections are shown. Secend row: $\ome$ of image in the first row. Third row: . }
%     \label{Fig:edge_regions}
% \end{figure}

% Subsection: Active Contour Models
\subsection{Active Contour Models}
\label{Subsec:active_contour}

The active contour model is one of the popular image segmentation and edge detection model. It is firstly introduced in~\cite{Kass:1988:Snakes}, named as snake. Let $I:\Omega\subset\mathbf{R}^2 \rightarrow [0,\,1]$ be a given image and $\calc:[0,\,1] \rightarrow \Omega$ be a parametric curve. The traditional snake finds contour which minimizes the energy functional 
\begin{align*}
    E_{snake}\left(\calc(s)\right) &= \int_{0}^{1} \frac{1}{2}\left( \alpha|\calc'(s)|^2 + \beta |\calc''(s)|^2 \right) \operatorname{d}\!s + E_{ext}(\calc(s)) \cm
\end{align*}
where $\alpha$ and $\beta$ are positive constants. The first term regulates deformation of the contour, and the last term attracts the contour to where the $E_{ext}$ is minimized. There is another energy minimization model GVF~\cite{Xu:1997:GVF} that proposes a vector field $\mathbf{v}=(u,\,v)$ defined in $\Omega$ which minimizes the energy functional
\begin{align*}
    E_{GV\!F}(\mathbf{v}) = \int\int \mu\left( u_x^2 + u_y^2 + v_x^2 + v_y^2 \right) + |\nabla{f}|^2|\mathbf{v} - \nabla{f}|^2 \operatorname{d}\!x\operatorname{d}\!y\cm
\end{align*}
where $\mu$ is a constant and $f$ is an edge map having larger values near the image edges. Two commonly mentioned drawbacks are that the methods are sensitive to the position of the initial contour, and hard to approach to the concave boundary with difficulty of parameter choice~\cite{Caselles:1997:GAC,Hahn:2010:GADF,Xu:1997:GVF}.

Meanwhile, some attempts have been made by focusing the fact: movement of the contour is performed by force applied in the normal direction at the point on the contour. An example is the balloon model~\cite{Cohen:1991:Balloon} which proposes force for contour evolution
\begin{align*}
    F(s) = k_1\mathbf{n}(s) - k\frac{\nabla\calc(s)}{|\nabla\calc(s)|}\cm
\end{align*}
where $k_1,\,k_2$ are constants and $\mathbf{n}(s)$ is the unit normal vector at the point $\mathcal{C}(s)$. This model is very simple and powerful, but too naive that it only proceeds in one direction according to the sign of $k_1$. The GAC~\cite{Caselles:1997:GAC} which is a contour that minimizes the following energy functional is proposed:
\begin{align}
    E_{GAC} = \int_0^1 {|\calc'(s)|g\left( |\nabla{I(\calc(s))}| \right)} \operatorname{d}\!s\cm \label{Eq:GAC_energe}
\end{align}
where $g$ is a function with $g(0) = 1$ and $g(r)\rightarrow 0$ as $r\rightarrow \infty$. The geodesic flow, derived from~\eqref{Eq:GAC_energe}, has level set formulation
\begin{align}
    \frac{\partial}{\partial{t}}\phi = g(I)\left( c + \kappa(\phi) \right)|\gphi| + \nabla{g}\cdot\gphi\cm \label{Eq:GAC_lvset}
\end{align}
where $c$ is a constant and $\kappa(\phi)$ is curvature of level set of the function $\phi$. Containing the relation between~\eqref{Eq:GAC_energe} and~\eqref{Eq:GAC_lvset}, equivalence relation between parametric and geometric active contour models can be achieved using the level set method~\cite{Xu:2000:ParamGeoAC}.

A contour evolution using the GADF~\eqref{Def:gadf} is proposed with level set formulation in~\cite{Hahn:2010:GADF}:
\begin{align}
    \frac{\partial}{\partial{t}}\phi(x,\,t) &= \mu\kappa(\phi)\left|\nabla\phi\right| + \chi_{\Omega_b}F_b|\nabla\phi| - \chi_{\oma}F_a\cdot\nabla\phi\cm \label{Eq:ac_gadf}\\
    \phi(x,\,0) &= \phi_0(x)\cm \nonumber
\end{align}
where $\mu$ is a constant, $\phi_0$ is the initial signed distance function (SDF), $\chi$ is a characteristic function of the subscripted set, $\oma$ is an edge-region, and $\Omega_b = \Omega\setminus\oma$. In~\cite{Hahn:2010:GADF}, $\oma=\delta_S(\ome)$ where $\delta_{S}(X)$ is a dilation of a set $X$ by a $3 \times 3$ square structuring element $S$ with origin at its center. By \eqref{Eq:ac_gadf}, the initial contour goes outward in $\Omega_b$ by a balloon force $F_b$ and reaches the edge following $F_a$ once it enters $\oma$. 

In Figure~\ref{Fig:evolution}, contours evolving by~\eqref{Eq:ac_gadf} with different initial positions are presented over time, using the edge-region in the bottom of~\ref{Fig:edge_regions_4} as $\oma$. As an initial contour evolves, unnecessary regions hinder the movement of the contour. There are two types of unnecessary regions; the regions having closed and non-closed shape. For the regions that are not forming the closed shape, the evolving contours can intrude that from ends, then encroach the region. On the other hands, if the contour faces a region with closed shape then, like a barrier, they block the movement of the contour and make it difficult to find the correct object boundary so that incorrect segmentation occurs.

% Figure: Edge regions outside the tooth boundaries
\begin{figure}
    \newcommand*{\wid}{3.4}%
    \newcommand*{\twid}{.27}%
    \centering
    \begin{subfigure}{ \twid \textwidth}
        \centering
        \includegraphics[height=\wid cm]{ ./Figures/edge_region/stt.png}
        \includegraphics[height=\wid cm]{ ./Figures/edge_region/nolight.png}
        \includegraphics[height=\wid cm]{ ./Figures/edge_region/light.png}
        \caption{\footnotesize Teeth images}
        %\caption{A synthetic image}
    \end{subfigure} 
    \begin{subfigure}{ \twid \textwidth}
        \centering
        \includegraphics[height=\wid cm]{ ./Figures/edge_region/stt_er.png}
        \includegraphics[height=\wid cm]{ ./Figures/edge_region/nolight_er.png}
        \includegraphics[height=\wid cm]{ ./Figures/edge_region/light_er.png}
        \caption{\footnotesize Edge-region candidate $\ome$}
    \end{subfigure}
    \begin{subfigure}{ \twid \textwidth}
        \centering
        \includegraphics[height=\wid cm]{ ./Figures/edge_region/st_Er_l.png}
        \includegraphics[height=\wid cm]{ ./Figures/edge_region/nolight_Er_l.png}
        \includegraphics[height=\wid cm]{ ./Figures/edge_region/light_Er_l.png}
        \caption{\footnotesize Edge-region $\oma$}
        \label{Fig:edge_regions_4}
    \end{subfigure}

    \caption{ (a) A synthetic image, a teeth image without light reflection, and a teeth image with light reflection~\cite{T001_21} are shown from top to bottom. (b) Edge-region candidates $\ome$ of the left images. (c) Edge-region $\oma$ is refined using component size and image intensity. }
    \label{Fig:edge_regions}
\end{figure}
% \begin{figure}
%     \newcommand*{\wid}{3.0}%
%     \newcommand*{\twid}{.23}%
%     \centering
%     \begin{subfigure}{ \twid \textwidth}
%         \centering
%         \includegraphics[height=\wid cm]{ ./Figures/edge_region/stt.png}
%         \includegraphics[height=\wid cm]{ ./Figures/edge_region/stt_er.png}
%         \includegraphics[height=\wid cm]{ ./Figures/edge_region/stt_Er_l.png}
%         \caption{}
%         %\caption{A synthetic image}
%     \end{subfigure} 
%     \begin{subfigure}{ \twid \textwidth}
%         \centering
%         \includegraphics[height=\wid cm]{ ./Figures/edge_region/nolight.png}
%         \includegraphics[height=\wid cm]{ ./Figures/edge_region/nolight_er.png}
%         \includegraphics[height=\wid cm]{ ./Figures/edge_region/nolight_Er_l.png}
%         \caption{}
%     \end{subfigure}
%     \begin{subfigure}{ \twid \textwidth}
%         \centering
%         \includegraphics[height=\wid cm]{ ./Figures/edge_region/light.png}
%         \includegraphics[height=\wid cm]{ ./Figures/edge_region/light_er.png}
%         \includegraphics[height=\wid cm]{ ./Figures/edge_region/light_Er_l.png}
%         \caption{}
%         \label{Fig:edge_regions_4}
%     \end{subfigure}

%     \caption{ First row: (a) A synthetic image, (b) teeth image without light reflections and (c) teeth image with light reflections are shown. Secend row: $\ome$ of image in the first row. Third row: . }
%     \label{Fig:edge_regions}
% \end{figure}

% Subsection: Light reflections in the teeth image
\subsection{Light reflections in teeth images}
\label{Subsec:light_reflection}

As discussed in previous Sections~\ref{Subsec:edge-regions}, \ref{Subsec:active_contour}, there are unnecessary regions in edge-regions that impedes the active contours. Teeth images have many problematic features which make unnecessary regions, but closed shape regions are mainly caused by strong light reflections on teeth and gums. The light reflections accompany strong intensity jumps and irregular shapes so that it is hard to remove the regions causing from the reflections using simple methods such as mentioned in~\ref{Subsec:edge-regions}.

One simple way to prevent these regions is to remove light reflections from the teeth image. There have been several studies to remove light reflections in a single image~\cite{SpecRemoval:2018,SpecRemoval:2020,SpecRemoval:2016,SpecRemoval:2015}. In these works, an image was decomposed into a non-reflective image and a reflection map in the image domain. These methods are based on the intensity histogram or assumption of a dichromatic image model~\cite{Shafer:1985:colorsep} using color information of the reflective region. However, in the case of teeth images, the colors of the light reflection and the tooth surface are almost the same due to the characteristics of human teeth. For this reason, when these methods are actually applied, parts of the tooth surface is also removed along with the light reflection. 

Although there was an attempt to remove the light reflection in teeth images using a single layer perceptron \cite{LeeRemoval:2010}, there is a limitation that it cannot be applied to a wide range of images because the perceptron has an extremely simple structure and learning is processed with only one image.

The most problematic thing is that even if light reflection is removed from the teeth image, individual tooth cannot be segmented immediately. This is because there are a lot of unnecessary edges besides the light reflection. Each is simple so it does not interfere with the movement of the contour, but if they stick together forming a large bunch or enclosing a region, it hinders the movement of the contour and becomes an obstacle to detecting the boundary of the tooth. In addition, when light reflection appears on the tooth boundary, the edge collapses when the light reflection is removed, so that the boundary becomes undetectable. Therefore, we seek a way to directly obtain only the edge-region near the tooth boundary, regardless of stains, noise or light reflection of the image. For this end, a supervised learning method with deep neural networks is considered.

% Section: Proposed Method
\section{Proposed Method}
\label{Sec:PM}
In this section, we describe the steps of the proposed individual tooth segmentation algorithm. It has the following three main steps and the whole process is shown in Figure~\ref{Fig:flowchart}:

\begin{itemize}
\setlength{\itemindent}{5mm}
    \item[STEP 1] Obtaining pseudo edge-regions from a neural network,
    % \item[STEP 2] Acquiring initial contours using morphological characteristics of teeth regions,
    \item[STEP 2] Segmentation using active contours using initial contours acquired from the pseudo edge-region,
    \item[STEP 3] Identification of tooth and non-tooth region.
\end{itemize}

% Figure: Algorithm flowchart
\begin{figure}
    \newcommand*{\wid}{3.4}%
    \newcommand*{\twid}{.27}%
    \centering
    \begin{subfigure}{ \twid \textwidth}
        \centering
        \includegraphics[height=\wid cm]{ ./Figures/edge_region/stt.png}
        \includegraphics[height=\wid cm]{ ./Figures/edge_region/nolight.png}
        \includegraphics[height=\wid cm]{ ./Figures/edge_region/light.png}
        \caption{\footnotesize Teeth images}
        %\caption{A synthetic image}
    \end{subfigure} 
    \begin{subfigure}{ \twid \textwidth}
        \centering
        \includegraphics[height=\wid cm]{ ./Figures/edge_region/stt_er.png}
        \includegraphics[height=\wid cm]{ ./Figures/edge_region/nolight_er.png}
        \includegraphics[height=\wid cm]{ ./Figures/edge_region/light_er.png}
        \caption{\footnotesize Edge-region candidate $\ome$}
    \end{subfigure}
    \begin{subfigure}{ \twid \textwidth}
        \centering
        \includegraphics[height=\wid cm]{ ./Figures/edge_region/st_Er_l.png}
        \includegraphics[height=\wid cm]{ ./Figures/edge_region/nolight_Er_l.png}
        \includegraphics[height=\wid cm]{ ./Figures/edge_region/light_Er_l.png}
        \caption{\footnotesize Edge-region $\oma$}
        \label{Fig:edge_regions_4}
    \end{subfigure}

    \caption{ (a) A synthetic image, a teeth image without light reflection, and a teeth image with light reflection~\cite{T001_21} are shown from top to bottom. (b) Edge-region candidates $\ome$ of the left images. (c) Edge-region $\oma$ is refined using component size and image intensity. }
    \label{Fig:edge_regions}
\end{figure}
% \begin{figure}
%     \newcommand*{\wid}{3.0}%
%     \newcommand*{\twid}{.23}%
%     \centering
%     \begin{subfigure}{ \twid \textwidth}
%         \centering
%         \includegraphics[height=\wid cm]{ ./Figures/edge_region/stt.png}
%         \includegraphics[height=\wid cm]{ ./Figures/edge_region/stt_er.png}
%         \includegraphics[height=\wid cm]{ ./Figures/edge_region/stt_Er_l.png}
%         \caption{}
%         %\caption{A synthetic image}
%     \end{subfigure} 
%     \begin{subfigure}{ \twid \textwidth}
%         \centering
%         \includegraphics[height=\wid cm]{ ./Figures/edge_region/nolight.png}
%         \includegraphics[height=\wid cm]{ ./Figures/edge_region/nolight_er.png}
%         \includegraphics[height=\wid cm]{ ./Figures/edge_region/nolight_Er_l.png}
%         \caption{}
%     \end{subfigure}
%     \begin{subfigure}{ \twid \textwidth}
%         \centering
%         \includegraphics[height=\wid cm]{ ./Figures/edge_region/light.png}
%         \includegraphics[height=\wid cm]{ ./Figures/edge_region/light_er.png}
%         \includegraphics[height=\wid cm]{ ./Figures/edge_region/light_Er_l.png}
%         \caption{}
%         \label{Fig:edge_regions_4}
%     \end{subfigure}

%     \caption{ First row: (a) A synthetic image, (b) teeth image without light reflections and (c) teeth image with light reflections are shown. Secend row: $\ome$ of image in the first row. Third row: . }
%     \label{Fig:edge_regions}
% \end{figure}
    
% Subsection: Obtaining pseudo edge-regions from a neural network
\subsection{Obtaining pseudo edge-region from a neural network}
\label{Subsec:pseudo_er}

As shown in Figure~\ref{Fig:flowchart}(a), we make labels for training data using GADF and manual selection. First, we manually select the connected components of $\ome$ intersecting the tooth boundary, then the label $Y\colon\Omega\subset\mathbf{R}^2\rightarrow\{0,\,1\}$ is defined as a binary image
\begin{align*}
    Y(x) =
    \begin{cases}
        1 & x\in\oms\cm\\
        0 & \mbox{otherwise}\cm
    \end{cases}
\end{align*}
where $\oms$ is obtained by dilating the regions manually selected from $\ome$ using an $\eta\times\eta$ square structuring element with the origin at its center. In Figure~\ref{Fig:labeling}, the labeling process is presented.

We consider a neural network consisting of the part of ResNeSt-50~\cite{Zhang:2020:ResNeSt} before the global average pooling as the encoder part and a custom upscaling module as the decoder part. The network takes a three channel image $X\colon\Omega\rightarrow\mathbf{R}^3$ as an input and produces an image $\hat{Y}\colon\Omega\rightarrow(0,\,1)$ as an output and is trained by minimizing the binary cross entropy (BCE) loss function \cite{Zhang:2018:BCE} of $Y$ and $\hat{Y}$. The entire network structure is shown in Appendix~\ref{App:neural_net}.

% Figure: Labeling images
\begin{figure}
    \newcommand*{\wid}{3.4}%
    \newcommand*{\twid}{.27}%
    \centering
    \begin{subfigure}{ \twid \textwidth}
        \centering
        \includegraphics[height=\wid cm]{ ./Figures/edge_region/stt.png}
        \includegraphics[height=\wid cm]{ ./Figures/edge_region/nolight.png}
        \includegraphics[height=\wid cm]{ ./Figures/edge_region/light.png}
        \caption{\footnotesize Teeth images}
        %\caption{A synthetic image}
    \end{subfigure} 
    \begin{subfigure}{ \twid \textwidth}
        \centering
        \includegraphics[height=\wid cm]{ ./Figures/edge_region/stt_er.png}
        \includegraphics[height=\wid cm]{ ./Figures/edge_region/nolight_er.png}
        \includegraphics[height=\wid cm]{ ./Figures/edge_region/light_er.png}
        \caption{\footnotesize Edge-region candidate $\ome$}
    \end{subfigure}
    \begin{subfigure}{ \twid \textwidth}
        \centering
        \includegraphics[height=\wid cm]{ ./Figures/edge_region/st_Er_l.png}
        \includegraphics[height=\wid cm]{ ./Figures/edge_region/nolight_Er_l.png}
        \includegraphics[height=\wid cm]{ ./Figures/edge_region/light_Er_l.png}
        \caption{\footnotesize Edge-region $\oma$}
        \label{Fig:edge_regions_4}
    \end{subfigure}

    \caption{ (a) A synthetic image, a teeth image without light reflection, and a teeth image with light reflection~\cite{T001_21} are shown from top to bottom. (b) Edge-region candidates $\ome$ of the left images. (c) Edge-region $\oma$ is refined using component size and image intensity. }
    \label{Fig:edge_regions}
\end{figure}
% \begin{figure}
%     \newcommand*{\wid}{3.0}%
%     \newcommand*{\twid}{.23}%
%     \centering
%     \begin{subfigure}{ \twid \textwidth}
%         \centering
%         \includegraphics[height=\wid cm]{ ./Figures/edge_region/stt.png}
%         \includegraphics[height=\wid cm]{ ./Figures/edge_region/stt_er.png}
%         \includegraphics[height=\wid cm]{ ./Figures/edge_region/stt_Er_l.png}
%         \caption{}
%         %\caption{A synthetic image}
%     \end{subfigure} 
%     \begin{subfigure}{ \twid \textwidth}
%         \centering
%         \includegraphics[height=\wid cm]{ ./Figures/edge_region/nolight.png}
%         \includegraphics[height=\wid cm]{ ./Figures/edge_region/nolight_er.png}
%         \includegraphics[height=\wid cm]{ ./Figures/edge_region/nolight_Er_l.png}
%         \caption{}
%     \end{subfigure}
%     \begin{subfigure}{ \twid \textwidth}
%         \centering
%         \includegraphics[height=\wid cm]{ ./Figures/edge_region/light.png}
%         \includegraphics[height=\wid cm]{ ./Figures/edge_region/light_er.png}
%         \includegraphics[height=\wid cm]{ ./Figures/edge_region/light_Er_l.png}
%         \caption{}
%         \label{Fig:edge_regions_4}
%     \end{subfigure}

%     \caption{ First row: (a) A synthetic image, (b) teeth image without light reflections and (c) teeth image with light reflections are shown. Secend row: $\ome$ of image in the first row. Third row: . }
%     \label{Fig:edge_regions}
% \end{figure}

The network output $\hat{Y}(x)$ can be thought as the probability map that an edge-region caused by the tooth boundary exists at $x$. We call
\begin{align}
    \PER = \left\{ x\in\Omega \mid \hat{Y}(x) > 0.5 \right\} \label{Def:omcc}
\end{align}
as a pseudo edge-region of $X$, and examples of pseudo edge-regions for teeth images are shown in the first two rows of Figure~\ref{Fig:pre_er}. One notable point is that the pseudo edge-regions obtained by the neural network look better than the regions obtained by the manual selection from $\ome$, i.e. the labels.

% Figure: pre_er
\begin{figure}
    \newcommand*{\wid}{3.4}%
    \newcommand*{\twid}{.27}%
    \centering
    \begin{subfigure}{ \twid \textwidth}
        \centering
        \includegraphics[height=\wid cm]{ ./Figures/edge_region/stt.png}
        \includegraphics[height=\wid cm]{ ./Figures/edge_region/nolight.png}
        \includegraphics[height=\wid cm]{ ./Figures/edge_region/light.png}
        \caption{\footnotesize Teeth images}
        %\caption{A synthetic image}
    \end{subfigure} 
    \begin{subfigure}{ \twid \textwidth}
        \centering
        \includegraphics[height=\wid cm]{ ./Figures/edge_region/stt_er.png}
        \includegraphics[height=\wid cm]{ ./Figures/edge_region/nolight_er.png}
        \includegraphics[height=\wid cm]{ ./Figures/edge_region/light_er.png}
        \caption{\footnotesize Edge-region candidate $\ome$}
    \end{subfigure}
    \begin{subfigure}{ \twid \textwidth}
        \centering
        \includegraphics[height=\wid cm]{ ./Figures/edge_region/st_Er_l.png}
        \includegraphics[height=\wid cm]{ ./Figures/edge_region/nolight_Er_l.png}
        \includegraphics[height=\wid cm]{ ./Figures/edge_region/light_Er_l.png}
        \caption{\footnotesize Edge-region $\oma$}
        \label{Fig:edge_regions_4}
    \end{subfigure}

    \caption{ (a) A synthetic image, a teeth image without light reflection, and a teeth image with light reflection~\cite{T001_21} are shown from top to bottom. (b) Edge-region candidates $\ome$ of the left images. (c) Edge-region $\oma$ is refined using component size and image intensity. }
    \label{Fig:edge_regions}
\end{figure}
% \begin{figure}
%     \newcommand*{\wid}{3.0}%
%     \newcommand*{\twid}{.23}%
%     \centering
%     \begin{subfigure}{ \twid \textwidth}
%         \centering
%         \includegraphics[height=\wid cm]{ ./Figures/edge_region/stt.png}
%         \includegraphics[height=\wid cm]{ ./Figures/edge_region/stt_er.png}
%         \includegraphics[height=\wid cm]{ ./Figures/edge_region/stt_Er_l.png}
%         \caption{}
%         %\caption{A synthetic image}
%     \end{subfigure} 
%     \begin{subfigure}{ \twid \textwidth}
%         \centering
%         \includegraphics[height=\wid cm]{ ./Figures/edge_region/nolight.png}
%         \includegraphics[height=\wid cm]{ ./Figures/edge_region/nolight_er.png}
%         \includegraphics[height=\wid cm]{ ./Figures/edge_region/nolight_Er_l.png}
%         \caption{}
%     \end{subfigure}
%     \begin{subfigure}{ \twid \textwidth}
%         \centering
%         \includegraphics[height=\wid cm]{ ./Figures/edge_region/light.png}
%         \includegraphics[height=\wid cm]{ ./Figures/edge_region/light_er.png}
%         \includegraphics[height=\wid cm]{ ./Figures/edge_region/light_Er_l.png}
%         \caption{}
%         \label{Fig:edge_regions_4}
%     \end{subfigure}

%     \caption{ First row: (a) A synthetic image, (b) teeth image without light reflections and (c) teeth image with light reflections are shown. Secend row: $\ome$ of image in the first row. Third row: . }
%     \label{Fig:edge_regions}
% \end{figure}

% Subsection: Acquiring initial contours and segmentation using active contours
\subsection{Active contour model and initial contours}
\label{Subsec:SegLevelset}

In this section, we propose active contour models formulated as level set method to acquire initial contours then segment regions. In the next two sections, we will discuss about initial contours and the region segmentation, sequentially. For the simplicity of the literature, we denote $\sdf(A)$ by a SDF which is negative on the set $A$ and constantly zero on $\partial A$.

\subsubsection{Acquiring initial contours}
\label{Subsec:initial_contours}

If the pseudo edge-region is perfectly matched with tooth boundaries, then all tooth regions are surrounded by thin closed curves, so that initial contours are given naturally by boundaries of the enclosed regions. However, pseudo edge-region is not perfect and contains many leakages, which mean breakage of the closed curve. Thus, multiple objects may merged into a single connected region and it is required to separate the merged objects from the region. When multiple objects are merged, the shape or orientation of the region is distinguished from the regions containing a single tooth. Actually, we do not need to think about such single tooth regions, because its boundaries can be initial contours. Hence we first identify the tooth regions from the other regions. 

For a connected region $X$, let $c$ be the center of mass and $\mathcal{I}$ be the 2 dimensional moment of inertia tensor about $c$ defined by
\begin{align*}
    \mathcal{I} &=
    \begin{bmatrix}
        I_{xx} & I_{xy}\\
        I_{yx} & I_{yy}
    \end{bmatrix}\cm
\end{align*}
where $x$, $y$ are coordinates of the vector from $c$ to each point of $X$, and for point mass $m$,
\begin{align*}
    I_{xx} = \int_{X} my^2 \cm\quad I_{yy} = \int_{X} mx^2\quad\mbox{and}\quad I_{xy} = I_{yx} = -\int_{X} mxy\pd
\end{align*}
Then the inertia tensor has two eigenvalues $\Lambda$ and $\lambda$ ($\Lambda > \lambda$). Let $v$ be a normalized eigenvector corresponding to $\lambda$. The methods listed below are criterions for identifying non-tooth regions: 
\begin{itemize}
    \setlength{\itemindent}{2mm}
    \item [(C--1)] If the center of mass $c$ is outside the region, the region is identified as not a tooth region.

    \item [(C--2)] If $\Lambda / \lambda > 2.5$ and $|v\cdot e_1| \ge \cos(\pi/4)$ for $e_1 = (1,\,0)$, the region is identified as not a tooth region.
    
    \item [(C--3)] Let $\kappa(s)$ be an Euclidean curvature on $s\in\partial{X}$. If 
    \begin{align*}
        \frac{1}{|\partial{X}|}\int_{\partial{X}} \mbox{sgn}(\kappa(s)) \operatorname{d}\!s < -0.2\cm
    \end{align*}
    then the region is identified as not a tooth region.
\end{itemize}
Let $\omt$ and $\omn$ be tooth and non-tooth regions identified by the above list, respectively. For $\omt$, there is nothing to do as mentioned above. For $\omn$, by taking all local minima of $\sdf(\omn)$, we can retain landmarks for each object in the merged region; see the third row of Figure~\ref{Fig:pre_er}. 

In order to acquire nicer initial contours, each landmark is enlarged as much as possible to contact pseudo edge-region while keeping its place. Let $\{L_i\}_{i=1}^K$ be distinct $K$ landmarks. For each level set function $\phi_i$, the competing balloon force $\fci$ is defined as
\begin{align}
    \fci(x,\,t) =
        -1 - \sum_{j\neq i}\min\left(\phi_j(x,\,t) - 1,\,0\right)\cm \label{Def:CompeteBalloon}
\end{align}
which is designed to inflate the zero level contour $\{\phi_i = 0\}$ until it meets $\{\phi_j = 0\}$ for any $j\neq i$, then stop. Here, a contour evolution using the competing balloon force is proposed
\begin{align}
    \frac{\partial}{\partial{t}}\phi_i(x,\,t) &= \mu\chi_{\Omega\setminus\PER} \kappa(\phi_i)\ngphii + \left(\chi_{\Omega\setminus\PER}\fci +\nu\chi_{\PER}\right)\ngphii\cm \label{Eq:evolv_init}\\
    \phi_i(x,\,0) &= \phi_{i}^L(x)\cm \nonumber
\end{align}
where $\mu$, $\nu$ are constants, $\kappa(\phi_i)$ is the curvature of $\phi_i$, $\PER$ is the pseudo edge-region, and $\phi_i^L = \sdf(L_i)$ for all $i=1,\,\ldots,\,K$. The first term is regularization term, and the second term makes initial contours move to $\partial\PER$, then stop. In order to resolve the over-segmentation, the standby process is introduced when applying the evolution by~\eqref{Eq:evolv_init}; until all the contours fall within a certain range of $\PER$, contours that touch each other are merged rather than stop by $\fci$. Resulting initial contours are shown in the last row of Figure~\ref{Fig:pre_er}.

\subsubsection{Segmentation using active contours}
\label{Subsec:segmentation}

We segment each region using multiple active contours initiated in the previous section. Let $\left\{\phi_{i,\,0}\right\}_{i=1}^{N}$ be a collection of level set functions which are steady-state solution of~\eqref{Eq:evolv_init}. Here, we are assuming that the tooth boundaries are contained in $\PER$. Since the contours are close to the $\partial\PER$, we push $\phi_{i,\,0}$ outward until each of them reaches the boundary of the tooth. For this end, we propose a contour evolution with a level set formulation:
\begin{align}
    \frac{\partial}{\partial{t}}\phi_i(x,\,t) &= \mu\kappa(\phi_i)\ngphii + \chi_{\ERC}\left( \fci + {\fsi} \right) \ngphii-\chi_{\ER}F_a\cdot \gphi \cm \label{Eq:proposed}\\
    \phi_i(x,\,0) &= \phi_{i,\,0}(x)\cm \nonumber
\end{align}
where $\mu$ is a constant, $\kappa(\phi_i)$ is the curvature of $\phi_i$. Here are three forces $F_a$,~$\fsi$ and $\fci$; $F_a$ is GADF defined in~\eqref{Def:gadf} and $\fci$ is the competing balloon force~\eqref{Def:CompeteBalloon}. The force $\fsi$ is called the statistically reinstating force~(SRF) proposed in~\cite{Park:2014:SRM}. It examines intensity distributions of local regions of a certain point and is decided as $1$ or $-1$ to push the contour so that $x$ belongs to a region with more similar intensity.

When a contour $\calc$ evolves by~\eqref{Eq:proposed}, the first term controls the smoothness of $\calc$, i.e., makes it round by acting opposite direction to where the contour is too banded or pointed, and the other terms control the movement of contours. When $\calc$ is in $\ER$, it is only affected by the third term. The third term attracts $\calc$ to where the GADF points, and stops. On the other hand, when $\calc$ is in $\ERC$, the contour is affected by the second term. It should be divided into two cases according to the relation between $\calc$ and other contours: Let a point $x$ be in $\calc$. In the case of $x$ is far from the other contours, $\fci=-1$ so that the second term of~\eqref{Eq:proposed} becomes ${\fsi} - 1$. If $\fsi(x)=-1$, then the second term becomes $-2$, so $\calc$ moves outward but if $\fsi(x)=1$, the second term becomes $0$, so $\calc$ stops. In the case of $x$ closes to another contour, then $\fci$ becomes $0$ and $\calc$ affects by $\fsi$. Thus the contour moves following the SRF~\cite{Park:2014:SRM}. In Figure~\ref{Fig:proposed}, we can see how contour moves differently under different situations.

% Figure: proposed evolution
\begin{figure}
    \newcommand*{\wid}{3.4}%
    \newcommand*{\twid}{.27}%
    \centering
    \begin{subfigure}{ \twid \textwidth}
        \centering
        \includegraphics[height=\wid cm]{ ./Figures/edge_region/stt.png}
        \includegraphics[height=\wid cm]{ ./Figures/edge_region/nolight.png}
        \includegraphics[height=\wid cm]{ ./Figures/edge_region/light.png}
        \caption{\footnotesize Teeth images}
        %\caption{A synthetic image}
    \end{subfigure} 
    \begin{subfigure}{ \twid \textwidth}
        \centering
        \includegraphics[height=\wid cm]{ ./Figures/edge_region/stt_er.png}
        \includegraphics[height=\wid cm]{ ./Figures/edge_region/nolight_er.png}
        \includegraphics[height=\wid cm]{ ./Figures/edge_region/light_er.png}
        \caption{\footnotesize Edge-region candidate $\ome$}
    \end{subfigure}
    \begin{subfigure}{ \twid \textwidth}
        \centering
        \includegraphics[height=\wid cm]{ ./Figures/edge_region/st_Er_l.png}
        \includegraphics[height=\wid cm]{ ./Figures/edge_region/nolight_Er_l.png}
        \includegraphics[height=\wid cm]{ ./Figures/edge_region/light_Er_l.png}
        \caption{\footnotesize Edge-region $\oma$}
        \label{Fig:edge_regions_4}
    \end{subfigure}

    \caption{ (a) A synthetic image, a teeth image without light reflection, and a teeth image with light reflection~\cite{T001_21} are shown from top to bottom. (b) Edge-region candidates $\ome$ of the left images. (c) Edge-region $\oma$ is refined using component size and image intensity. }
    \label{Fig:edge_regions}
\end{figure}
% \begin{figure}
%     \newcommand*{\wid}{3.0}%
%     \newcommand*{\twid}{.23}%
%     \centering
%     \begin{subfigure}{ \twid \textwidth}
%         \centering
%         \includegraphics[height=\wid cm]{ ./Figures/edge_region/stt.png}
%         \includegraphics[height=\wid cm]{ ./Figures/edge_region/stt_er.png}
%         \includegraphics[height=\wid cm]{ ./Figures/edge_region/stt_Er_l.png}
%         \caption{}
%         %\caption{A synthetic image}
%     \end{subfigure} 
%     \begin{subfigure}{ \twid \textwidth}
%         \centering
%         \includegraphics[height=\wid cm]{ ./Figures/edge_region/nolight.png}
%         \includegraphics[height=\wid cm]{ ./Figures/edge_region/nolight_er.png}
%         \includegraphics[height=\wid cm]{ ./Figures/edge_region/nolight_Er_l.png}
%         \caption{}
%     \end{subfigure}
%     \begin{subfigure}{ \twid \textwidth}
%         \centering
%         \includegraphics[height=\wid cm]{ ./Figures/edge_region/light.png}
%         \includegraphics[height=\wid cm]{ ./Figures/edge_region/light_er.png}
%         \includegraphics[height=\wid cm]{ ./Figures/edge_region/light_Er_l.png}
%         \caption{}
%         \label{Fig:edge_regions_4}
%     \end{subfigure}

%     \caption{ First row: (a) A synthetic image, (b) teeth image without light reflections and (c) teeth image with light reflections are shown. Secend row: $\ome$ of image in the first row. Third row: . }
%     \label{Fig:edge_regions}
% \end{figure}

% Subsection: Teeth and non-teeth region classification
\subsection{Identification of tooth and non-tooth regions}
\label{Subsec:regClass}

As a final step, the identification of the segmented regions remains. Since our goal is to segment each individual tooth, we need to identify tooth and non-tooth regions from the result in Section~\ref{Subsec:SegLevelset}. In a human teeth image, tooth and non-tooth regions can be distinguished by shape and color. Teeth are relatively white and convex in shape, thus non-tooth regions can be easily removed. 

In Section~\ref{Subsec:initial_contours}, we identified tooth and non-tooth regions using morphological characteristics of the regions. But it is not always possible to identify regions using morphological characteristics. In order to exclude the non-tooth regions by considering not only morphologies but also color information, the method~(C-4) below is applied following the methods~(C-1)--(C-3) in Section~\ref{Subsec:initial_contours}: The CIE-Lab color space~\cite{Zeileis:2009:escaping} is designed to express the human vision system. It expresses color with three values ​​$L$, $a$, and $b$: $L$ represents lightness, and $a$ and $b$ represent four colors red, green, blue, and yellow. Let the image $I$ be embedded into the CIE-Lab color space denoted by $I_{Lab}(x)=[L(x),\,a(x),\,b(x)]$, and let $x^*\in\Omega$ be a point such that
\begin{align*}
    x^* = \argmax a(x)\pd
\end{align*}
Then, $x^*$ is the point at which $I(x^*)$ has the most red color.
\begin{itemize}
    \setlength{\itemindent}{2mm}
    \item [(C--4)] If $I_{Lab}$ values on the region is $k$-means clustered together with $I_{Lab}(x^*)$ among all the regions sharing the $x$-coordinate with the region, then the region is identified as not a tooth region.
\end{itemize}

% Section: Experimental results
\section{Experimental results and numerical aspect}
\label{Sec:result}

\textbf{Parameters and Numerical schemes.} In Section~\ref{Subsec:pseudo_er}, $\eta=\lfloor|\Omega| / 600 + 1/2\rfloor$ and in the list of Section~\ref{Subsec:initial_contours}, the point mass $m=1$. In all experiments, the parameters $\mu=2$ and $\nu=0.5$ in~\eqref{Eq:evolv_init}, and $\mu=2$ in~\eqref{Eq:proposed}. In addition, when~\eqref{Eq:evolv_init} is applied, $\PER$ is morphologically closed to improve the condition of the edge-region by a $\theta \times \theta$ square structuring element with origin at its center where $\theta=\max\{\lfloor \omega / 1.5 + 0.5 \rfloor, 1\}$ and $\omega$ is the average thickness of $\PER$. 

Overall, the finite difference scheme is applied to all derivatives. For contour evolutions, the explicit Euler method is used, and the time increment was set as 0.3 in all cases. While performing the contour evolutions, the level set function is frequently reinitialized by the method in~\cite{SUSSMAN:1994}. 

\noindent\textbf{Data augmentation.} To get more diverse training data, augmentation is applied. In each epoch, images are resized with randomly sampled height in $[256,\, 512]$ by keeping vertical and horizontal ratio. After that, a $256\times256$ patch is randomly cropped from either the resized image or its horizontal flip~\cite{He:2015:ResNet,Krizhevsky:2012:ImageNet}. For each cropping, the additional augmentation in the following list is randomly applied with the probability $0.5$. The parameters are uniformly selected within the given ranges:
\begin{itemize}
    \item Gaussian smoothing with $\sigma\in[0.25,\,0.75]$,
    \item Adding Gaussian noise with $\sigma\in[0.005,\,0.015]$,
    \item Gamma correction with $\gamma\in[0.5,\,2]$.
\end{itemize}

\noindent\textbf{Training.} The training dataset consists of $46$ images obtained from web searches using various search terms~\cite{T000,T001,T002,T003,T004,T005,T006,T007,T008,T009,T010,T011,T012,T013,T017,T018,T019,T020}. All teeth images contain light reflections on the surface of teeth, gums and other parts.

The neural network is trained by minimizing the BCE loss function \cite{Zhang:2018:BCE} using the Adam optimizer \cite{Kingma:2017}. The exponential decay rates for the moment of Adam optimizer are set to $(\beta_1,\,\beta_2)=(0.9,\,0.999)$ and the learning rate is $0.005$. Network is trained for $10{,}000$ epochs with $46$ iterations for each epoch, and the parameters after the last epoch are selected. 

It seems that the number of epochs is excessive compared to generally selected epoch number, but it is not. For each image, there are minimum $16$ image variant we can get with data augmentation, even if patch cropping and resizing are excluded. Since the parameters of each data augmentation are randomly selected for each time as well as randomly selected patches from random size images are used, it can be thought of as compared with typical $100$ epochs.

\noindent\textbf{Results.} In Figure~\ref{Fig:results}, several teeth images with its segmentation results for existing and proposed methods are presented. 

The results of model-based method~\cite{Na:2014LteethMorph} is in Figure~\ref{Fig:na2014}. The parameters that cannot be obtained in~\cite{Na:2014LteethMorph} are selected in our best effort. Many bad parts appear in the results in~\ref{Fig:na2014} such as captured light reflections or mis-segmented tooth boundaries. 

The Figures~\ref{Fig:mrcnn} and~\ref{Fig:mrcnn_w} shows results using the methods based on the Mask R-CNN~\cite{He:2018:MRCNN,Pham:2020,Zhu:2020:teethMaskrcnn}. For the training, we use the same training data as in this paper, but the types of label are different. The Mask R-CNN directly requires individual tooth regions as label, and these are generated through a well-known labeling tool, the labelme~\cite{Russell:2007:labelme}. Data augmentation is exactly the same as the papers. As shown in the figures, methods used in the last three columns of~\ref{Fig:results} shows prominent results. While the results from Mask R-CMM look similar, our results are distinct from the two. Here, some differences can be observed: Since the results using Mask R-CNN are pixel-based segmentation, the segmenting line appears stepwise along the pixel as well as gaps occur with the actual tooth. On the contrary, because our method uses level sets, the segmenting line is smooth and fitting the actual tooth. Also, compared to our results, the results of Mask R-CNN seem to keep the region of each instance more convex. For this reason, it looks evenly segments the shape of each tooth and having more smooth boundaries. However, for the same reason, it has the disadvantage of finding wrong boundaries where sharp or narrow region occurs. Lastly, the Mask R-CNN shows relatively poor capturing for the outermost teeth of the image comparing with our method.

Quantitative measurement is impossible since there is no ground truth, but compared to the results of the existing methods, it can be seen that the tooth regions are neatly segmented. In particular, only the tooth regions are segmented in the teeth images where the lips and the face part exist.

% Figure: Edge region candidate and results
\begin{figure}
    \newcommand*{\wid}{3.4}%
    \newcommand*{\twid}{.27}%
    \centering
    \begin{subfigure}{ \twid \textwidth}
        \centering
        \includegraphics[height=\wid cm]{ ./Figures/edge_region/stt.png}
        \includegraphics[height=\wid cm]{ ./Figures/edge_region/nolight.png}
        \includegraphics[height=\wid cm]{ ./Figures/edge_region/light.png}
        \caption{\footnotesize Teeth images}
        %\caption{A synthetic image}
    \end{subfigure} 
    \begin{subfigure}{ \twid \textwidth}
        \centering
        \includegraphics[height=\wid cm]{ ./Figures/edge_region/stt_er.png}
        \includegraphics[height=\wid cm]{ ./Figures/edge_region/nolight_er.png}
        \includegraphics[height=\wid cm]{ ./Figures/edge_region/light_er.png}
        \caption{\footnotesize Edge-region candidate $\ome$}
    \end{subfigure}
    \begin{subfigure}{ \twid \textwidth}
        \centering
        \includegraphics[height=\wid cm]{ ./Figures/edge_region/st_Er_l.png}
        \includegraphics[height=\wid cm]{ ./Figures/edge_region/nolight_Er_l.png}
        \includegraphics[height=\wid cm]{ ./Figures/edge_region/light_Er_l.png}
        \caption{\footnotesize Edge-region $\oma$}
        \label{Fig:edge_regions_4}
    \end{subfigure}

    \caption{ (a) A synthetic image, a teeth image without light reflection, and a teeth image with light reflection~\cite{T001_21} are shown from top to bottom. (b) Edge-region candidates $\ome$ of the left images. (c) Edge-region $\oma$ is refined using component size and image intensity. }
    \label{Fig:edge_regions}
\end{figure}
% \begin{figure}
%     \newcommand*{\wid}{3.0}%
%     \newcommand*{\twid}{.23}%
%     \centering
%     \begin{subfigure}{ \twid \textwidth}
%         \centering
%         \includegraphics[height=\wid cm]{ ./Figures/edge_region/stt.png}
%         \includegraphics[height=\wid cm]{ ./Figures/edge_region/stt_er.png}
%         \includegraphics[height=\wid cm]{ ./Figures/edge_region/stt_Er_l.png}
%         \caption{}
%         %\caption{A synthetic image}
%     \end{subfigure} 
%     \begin{subfigure}{ \twid \textwidth}
%         \centering
%         \includegraphics[height=\wid cm]{ ./Figures/edge_region/nolight.png}
%         \includegraphics[height=\wid cm]{ ./Figures/edge_region/nolight_er.png}
%         \includegraphics[height=\wid cm]{ ./Figures/edge_region/nolight_Er_l.png}
%         \caption{}
%     \end{subfigure}
%     \begin{subfigure}{ \twid \textwidth}
%         \centering
%         \includegraphics[height=\wid cm]{ ./Figures/edge_region/light.png}
%         \includegraphics[height=\wid cm]{ ./Figures/edge_region/light_er.png}
%         \includegraphics[height=\wid cm]{ ./Figures/edge_region/light_Er_l.png}
%         \caption{}
%         \label{Fig:edge_regions_4}
%     \end{subfigure}

%     \caption{ First row: (a) A synthetic image, (b) teeth image without light reflections and (c) teeth image with light reflections are shown. Secend row: $\ome$ of image in the first row. Third row: . }
%     \label{Fig:edge_regions}
% \end{figure}

% Section: Conclusion
\section{Conclusion}
\label{Sec:Conclusion}

In this paper, we proposed a method for individual tooth segmentation in a human teeth image. Including the strong light reflection, there are several obstacles that hinder the segmentation process. We solved these problem by obtaining pseudo edge-region using a deep neural network with supervised learning. In the labeling process, model-based methods were used to reduce time consumption and cost. In addition, we proposed contour evolutions to acquire initial contours from the pseudo edge-region, then to find the boundary of the tooth using different forces depending on the situation. Finally, by identifying tooth and non-tooth regions from the segmented result, we achieved the individual tooth segmentation a in human teeth image. 

Among the existing methods, model-based methods show insufficient performance. Although the supervised learning method using Mask R--CNN has similar performance to ours, it is difficult to label the data for training. This work is significant in that it proposed an algorithm that can easily prepare training data while performing well through a combination of model-based methods and a supervised learning method. It is noted that the combination of the two methods produced a synergistic effect; model-based methods streamline the labeling process for training deep neural networks and the networks simplify the problem by inferring edge-regions. 

The individual tooth segmentation can be importantly used for 3D teeth model reconstruction and dental forensic field. We hope that this study will be of great help in reconstructing a more sophisticated 3D teeth model from a 2D teeth image, and also in consisting more effective dental forensic system.

% Section: Acknowledgement
% \section*{Acknowledgement}

%% Appendix
\appendix
\section{Structure of neural network}
\label{App:neural_net}

% Figure: Network structure
\begin{subfigure}[]{1\textwidth}
    \newcommand*{\al}{0.3}%
    \newcommand*{\aw}{1.mm}%
    \newcommand*{\hei}{1.1 cm}%
    \newcommand*{\flr}{3.5cm}%
    \newcommand*{\wgap}{1cm}%
    \newcommand*{\init}{0}%
    \newcommand*{\wwa}{.2}%
    \newcommand*{\wwb}{.3}%
    \newcommand*{\wwc}{.4}%
    \newcommand*{\wwd}{.5}%
    \newcommand*{\wwe}{.6}%
    \newcommand*{\wwf}{.7}%
    \newcommand*{\wwg}{.9}%

    \newcommand*{\hha}{1}%
    \newcommand*{\hhb}{.9}%
    \newcommand*{\hhc}{.8}%
    \newcommand*{\hhd}{.6}%
    \newcommand*{\hhe}{.4}%
    \newcommand*{\hhf}{.2}%
    \newcommand*{\hhg}{.2}%
    \centering
    \begin{tikzpicture}[]
        \footnotesize
        \node (x0) at (0,0) [draw=none,minimum width=1mm,minimum height=\hei, fill=blue!50!white,label=above:{$3$}] {};
        \draw [-{Triangle[scale=.5]},line width=\aw,color=red!70!black] (x0.south) to ++(0,-\al);

        \node (x1) at ($(x0) + (0,-\hei-\al cm - .3 cm)$) [draw=none,minimum width=\wwa cm,minimum height=\hhb*\hei, fill=blue!50!white,label=above:{$32$}] {}; 
        \draw [-{Triangle[scale=.5]},line width=\aw,color=orange!80!white] (x1.east) -- ++(\al, 0);

        \node (x2) at ($(x1) + (\al+ \wwa/2 + \wwa/2,0)$) [draw=none,minimum width=\wwa cm,minimum height=\hhb*\hei, fill=blue!50!white,label=above:{$32$}] {};
        \draw [-{Triangle[scale=.5]},line width=\aw,color=orange!80!white] (x2.east) -- ++(\al, 0);
        
        \node (x3) at ($(x2) + (\al+\wwa/2 + \wwb/2,0)$) [draw=none,minimum width=\wwb cm,minimum height=\hhb*\hei, fill=blue!50!white,label=above:{$64$}] {};
        \draw [-{Triangle[scale=.5]},line width=\aw,color=yellow!80!black] (x3.south) to ++(0,-\al);
        
        \node (x4) at ($(x3) + (0,-\hhb*\hei-\al cm - .3 cm)$) [draw=none,minimum width=\wwb cm,minimum height=\hhc*\hei, fill=blue!50!white,label=above:{$64$}] {};
        \draw [-{Triangle[scale=.5]},line width=\aw,color=green!80!black] (x4.east) -- ++(\al, 0);
        
        \node (x5) at ($(x4) + (\al+\wwb/2 + \wwd/2,0)$) [draw=none,minimum width=\wwd cm,minimum height=\hhc*\hei, fill=blue!50!white,label=above:{$256$}] {};
        \draw [-{Triangle[scale=.5]},line width=\aw,color=green!80!black] (x5.east) -- node [above, color=black] {$\times 2$} ++(1.5*\al, 0);

        \node (x6) at ($(x5) + (1.5*\al+\wwd/2 + \wwd/2,0)$) [draw=none,minimum width=\wwd cm,minimum height=\hhc*\hei, fill=blue!50!white,label=above:{$256$}] {};
        \draw [-{Triangle[scale=.5]},line width=\aw,color=cyan!90!green] (x6.south) to  ++(0,-\al) {};

        \node (x7) at ($(x6) + (0,-\hhc*\hei-\al cm - .3 cm)$) [draw=none,minimum width=\wwe cm,minimum height=\hhd*\hei, fill=blue!50!white,label=above:{$512$}] {};
        \draw [-{Triangle[scale=.5]},line width=\aw,color=cyan!90!green] (x7.east) to node [above, color=black] {$\times 3$} ++(1.5*\al,0) {};

        \node (x8) at ($(x7) + (1.5*\al+\wwe/2 + \wwe/2,0)$) [draw=none,minimum width=\wwe cm,minimum height=\hhd*\hei, fill=blue!50!white,label=above:{$512$}] {};
        \draw [-{Triangle[scale=.5]},line width=\aw,color=blue!60!black] (x8.south) to  ++(0,-\al) {};

        \node (x9) at ($(x8) + (0,-\hhd*\hei-\al cm - .3 cm)$) [draw=none,minimum width=\wwf cm,minimum height=\hhe*\hei, fill=blue!50!white,label=above:{$1024$}] {};
        \draw [-{Triangle[scale=.5]},line width=\aw,color=blue!60!black] (x9.east) to node [above,color=black] {$\times 5$} ++(1.5*\al,0) {};

        \node (x10) at ($(x9) + (1.5*\al+\wwf/2 + \wwf/2,0)$) [draw=none,minimum width=\wwf cm,minimum height=\hhe*\hei, fill=blue!50!white,label=above:{$1024$}] {};
        \draw [-{Triangle[scale=.5]},line width=\aw,color=violet!90!black] (x10.south) to  ++(0,-\al) {};

        \node (x11) at ($(x10) + (0,-\hhe*\hei-\al cm - .3 cm)$) [draw=none,minimum width=\wwg cm,minimum height=\hhf*\hei, fill=blue!50!white,label=above:{$2048$}] {};
        \draw [-{Triangle[scale=.5]},line width=\aw,color=violet!90!black] (x11.east) to node [above, color=black] {$\times 2$} ++(1.5*\al,0) {};
        
        \node (x12) at ($(x11) + (1.5*\al+\wwg/2 + \wwg/2,0)$) [draw=none,minimum width=\wwg cm,minimum height=\hhf*\hei, fill=blue!50!white,label=above:{$2048$}] {};
        
        % latent variable
        \node (x13) at ($(x12) + (1.5*\al+\wwg/2 + \wwg/2,0)$) [draw=none,minimum width=\wwg cm,minimum height=2mm, fill=blue!50!white,label=above:{$2048$}] {};
        \draw [line width=\aw] (x12.east) to (x13.west) {};
        
        \node (x14) at ($(x13) + (0,\hhe*\hei+\al cm + .3 cm)$) [draw=none,minimum width=\wwg cm,minimum height=\hhe*\hei, fill=blue!50!white,label=above:{$2048$}] {};
        \draw [{Triangle[scale=.5]}-,line width=\aw,color=lime!50!black] (x14.south) to ++(0,-\al);
        
        \node (x15) at ($(x14) + (1.5*\al + \wwg/2 + \wwf/2,0)$) [draw=none,minimum width=\wwf cm,minimum height=\hhe*\hei, fill=blue!50!white,label=above:{$1024$}] {};
        \draw [-{Triangle[scale=.5]},line width=\aw,color=orange!80!white] (x14.east) to node [above, color=black] {$\times 2$} (x15.west) {};

        \node (x16) at ($(x15) + (0,\hhd*\hei+\al cm + .3 cm)$) [draw=none,minimum width=\wwf cm,minimum height=\hhd*\hei, fill=blue!50!white,label=above:{$1024$}] {};
        \draw [{Triangle[scale=.5]}-,line width=\aw,color=lime!50!black] (x16.south) to ++(0,-\al);
        
        \node (x17) at ($(x16) + (1.5*\al + \wwe/2 + \wwf/2,0)$) [draw=none,minimum width=\wwe cm,minimum height=\hhd*\hei, fill=blue!50!white,label=above:{$512$}] {};
        \draw [-{Triangle[scale=.5]},line width=\aw,color=orange!80!white] (x16.east) to node [above, color=black] {$\times 2$} (x17.west) {};

        \node (x18) at ($(x17) + (0,\hhc*\hei+\al cm + .3 cm)$) [draw=none,minimum width=\wwe cm,minimum height=\hhc*\hei, fill=blue!50!white,label=above:{$512$}] {};
        \draw [{Triangle[scale=.5]}-,line width=\aw,color=lime!50!black] (x18.south) to ++(0,-\al);
        
        \node (x19) at ($(x18) + (1.5*\al + \wwd/2 + \wwe/2,0)$) [draw=none,minimum width=\wwd cm,minimum height=\hhc*\hei, fill=blue!50!white,label=above:{$256$}] {};
        \draw [-{Triangle[scale=.5]},line width=\aw,color=orange!80!white] (x18.east) to node [above, color=black] {$\times 2$} (x19.west) {};

        \node (x20) at ($(x19) + (0,\hhb*\hei+\al cm + .3 cm)$) [draw=none,minimum width=\wwd cm,minimum height=\hhb*\hei, fill=blue!50!white,label=above:{$256$}] {};
        \draw [{Triangle[scale=.5]}-,line width=\aw,color=red!40!white] (x20.south) to ++(0,-\al);
        
        \node (x21) at ($(x20) + (1.5*\al + \wwd/2 + \wwc/2,0)$) [draw=none,minimum width=\wwc cm,minimum height=\hhb*\hei, fill=blue!50!white,label=above:{$128$}] {};
        \draw [-{Triangle[scale=.5]},line width=\aw,color=orange!80!white] (x20.east) to node [above, color=black] {$\times 2$} (x21.west) {};

        \node (x22) at ($(x21) + (0,\hei+\al cm + .3 cm)$) [draw=none,minimum width=\wwc cm,minimum height=\hei, fill=blue!50!white,label=above:{$128$}] {};
        \draw [{Triangle[scale=.5]}-,line width=\aw,color=red!40!white] (x22.south) to ++(0,-\al);
        
        \node (x23) at ($(x22) + (1.5*\al + \wwb/2 + \wwc/2,0)$) [draw=none,minimum width=\wwb cm,minimum height=\hei, fill=blue!50!white,label=above:{$64$}] {};
        \draw [-{Triangle[scale=.5]},line width=\aw,color=orange!80!white] (x22.east) to node [above, color=black] {$\times 2$} (x23.west) {};
        
        \node (x24) at ($(x23) + (\al + \wwb/2 + .1,0)$) [draw=none,minimum width=.1 cm,minimum height=\hei, fill=blue!50!white,label=above:{$1$}] {};
        \draw [-{Arc Barb[scale=.5]},line width=\aw,,color=red] (x23.east) to (x24.west) {};
    \end{tikzpicture}
    \caption{Network structure}
\end{subfigure}\\

\vspace{1mm}

\begin{subfigure}[]{.65\textwidth}
    \newcommand*{\hg}{.8}%
    \newcommand*{\wid}{.5}%
    \newcommand*{\hei}{1.3}%
    \newcommand*{\init}{0}%
    \newcommand*{\aw}{1mm}%
    \centering
    \begin{tikzpicture}
        \scriptsize
        \node (ipt) at (0,0) [rotate=90,draw=none,minimum width=\hei cm,minimum height=\wid cm, fill=orange!30!white, label=right:{$c$}, label={[xshift=-.7em, yshift=-.4em, rotate=90]}] {\footnotesize };
        
        \node (x1) at ($(ipt) + (\hg,0)$) [rotate=90,draw=none,minimum width=\hei cm,minimum height=\wid cm, fill=orange!30!white,  label=right:{$m$}, label={[xshift=-.7em, yshift=-.4em, rotate=90]}] {};

        \node (x21) at ($(x1) + (\hg,-1.5)$) [rotate=90,draw=none,minimum width=\hei cm,minimum height=\wid cm, fill=orange!30!white, label=right:{$m$}, label={[xshift=-.7em, yshift=-.4em, rotate=90]}] {};

        \node (x22) at ($(x1) + (\hg,1.5)$) [rotate=90,draw=none,minimum width=\hei cm,minimum height=\wid cm, fill=orange!30!white, label=right:{$m$}, label={[xshift=-.7em, yshift=-.4em, rotate=90]}] {};

        \node (sum1) at ($(x1) + (2*\hg,0)$) [circle, draw, very thick,minimum width=.1 cm,minimum height=.1 cm,text height=.15 cm, fill=none] {\footnotesize $+$};

        % \node (x3) at ($(x1) + (3*\hg,0)$) [rotate=90,draw=none,minimum width=\hei cm,minimum height=\wid cm, fill=orange!30!white, label=right:{$m$}, label={[xshift=-.7em, yshift=-.4em, rotate=90]}] {};

        \node (x4) at ($(x1) + (3*\hg,0)$) [rotate=90,draw=none,minimum width=.1 cm,minimum height=\wid cm, fill=orange!30!white, label=right:{$m$}, label={[xshift=-.7em, yshift=-.4em, rotate=90]}] {};

        \node (x5) at ($(x4) + (\hg,0)$) [rotate=90,draw=none,minimum width=.1 cm,minimum height=.7*\wid cm, fill=orange!30!white, label={[xshift=2.1em, yshift=.8em] $m/2$}, label={[xshift=-.7em, yshift=-.4em, rotate=90]}] {};

        \node (x61) at ($(x5) + (0,-.9)$) [rotate=90,draw=none,minimum width=.1 cm,minimum height=\wid cm, fill=orange!30!white, label=right:{$m$}, label={[xshift=-.7em, yshift=-.4em, rotate=90]}] {};

        \node (x62) at ($(x5) + (0,.9)$) [rotate=90,draw=none,minimum width=.1 cm,minimum height=\wid cm, fill=orange!30!white, label=right:{$m$}, label={[xshift=-.7em, yshift=-.4em, rotate=90]}] {};

        \node (prod11) at ($(x61) + (0,-1.1)$) [circle, draw, very thick,minimum width=.1 cm,minimum height=.1 cm, fill=none] {\footnotesize $\times$};
        \node (prod12) at ($(x62) + (0,1.1)$) [circle, draw, very thick,minimum width=.01 cm,minimum height=.01 cm,fill=none] {\scriptsize $\times$};

        \node (sum2) at ($(x5) + (\hg,0)$) [circle, draw, very thick,minimum width=.1 cm,minimum height=.1 cm,text height=.15 cm, fill=none] {\footnotesize $+$};

        \node (x7) at ($(x5) + (2*\hg,0)$) [rotate=90,draw=none,minimum width=\hei cm,minimum height=\wid cm, fill=orange!30!white, label=right:{$4m$}, label={[xshift=-.7em, yshift=-.4em, rotate=90]}] {};

        \node (sum3) at ($(x7) + (1*\hg,0)$) [circle, draw, very thick,minimum width=.1 cm,minimum height=.1 cm,text height=.15 cm, fill=none] {\footnotesize $+$};

        \node (x8) at ($(x7) + (2*\hg,0)$) [rotate=90,draw=none,minimum width=\hei cm,minimum height=\wid cm, fill=orange!30!white, label=right:{$4m$}, label={[xshift=-.7em, yshift=-.4em, rotate=90]}] {};

        \draw [-{Triangle[scale=.5]},line width=\aw,color=red] (ipt.south) to (x1.north) {};
        % \draw [-{Triangle[scale=.5]},line width=.2mm,color=black,dashed] (x1.south) to (x1.north) {};
        \draw [-{Triangle[scale=.5]},line width=\aw,color=orange!80!white] ($(x1.south) + (0,-.5)$) to (x21.north) {};
        \draw [-{Triangle[scale=.5]},line width=\aw,color=orange!80!white] ($(x1.south) + (0,.5)$) to (x22.north) {};
        % \draw [-{Triangle[scale=.5]},line width=\aw] (sum1.east) to (x3.north) {};
        \draw [-{Triangle[scale=.5]},line width=\aw] (sum1.east) to (x4.north) {};
        \draw [-{Triangle[scale=.5]},line width=\aw,color=magenta] (x4.south) to (x5.north) {};
        \draw [-{Bar[scale=.5]},line width=.5mm,color=magenta] (x5.west) to ++ (0,-.4);
        \draw [{Bar[scale=.5]}-,line width=.5mm,color=magenta] (x62.west) to ++ (0,-.4);
        \draw [-{Triangle[scale=.5]},line width=\aw,color=red] (sum2.east) to (x7);
        
        \draw [-{Triangle[scale=.5]},line width=.2mm] (x21.south) to (sum1) {};
        \draw [-{Triangle[scale=.5]},line width=.2mm] (x22.south) to (sum1) {};
        \draw [-{Triangle[scale=.5]},line width=.2mm] ($(x21.south) + (0,-.5)$) to (prod11);
        \draw [-{Triangle[scale=.5]},line width=.2mm] ($(x22.south) + (0,.5)$) to (prod12);
        \draw [{Triangle[scale=.5]}-,line width=.2mm] (prod11.north) to (x61.west);
        \draw [{Triangle[scale=.5]}-,line width=.2mm] (prod12.south) to ++(0,-.4);
        \draw [-{Triangle[scale=.5]},line width=.2mm] (prod11.east) to (sum2);
        \draw [-{Triangle[scale=.5]},line width=.2mm] (prod12.east) to (sum2);
        \draw [-{Triangle[scale=.5]},line width=.2mm] (prod12.east) to (sum2);
        \draw [-{Triangle[scale=.5]},line width=.2mm] (x7.south) to (sum3);
        \draw [dashed,line width=.25mm] (x1.west) -- ($(x21.west) + (0,-1cm)$) -- ($(x7.west) + (0,-2.5cm)$) -- (sum3.south) [-{Triangle[scale=.5]}];
        \draw [-{Triangle[scale=.5]},line width=.2mm] (sum3.east) to (x8.north);
    \end{tikzpicture}
    \caption{ResNeSt block with mid--channel size $m$ (RNS--$m$)}
\end{subfigure}
\begin{subfigure}[]{.3\textwidth}
    \newcommand*{\hg}{1.1}%
    \newcommand*{\wid}{.5}%
    \newcommand*{\hei}{2}%
    \newcommand*{\init}{-1}%
    \newcommand*{\aw}{.9mm}%
    \newcommand*{\al}{.5}%
    \centering
    \begin{tikzpicture}
        \footnotesize
        \matrix [draw,below left, thick] at (current bounding box) {
            \draw [line width=\aw,color=red!70!black] (\init,0) to ++(\al,0) node[right, color=black] {$3\times 3$ conv, $/2$, $+1$};\\

            \draw [line width=\aw,color=orange!80!white] (\init,0) to ++(\al,0) node[right, color=black] {$3\times 3$ conv, $+1$};\\

            \draw [line width=\aw,color=yellow!80!black] (\init,0) to ++(\al,0) node[right, color=black] {$3\times 3$ maxpool, $/2$, $+1$};\\

            \draw [line width=\aw,color=green!80!black] (\init,0) to ++(\al,0) node[right, color=black] {RNS--$64$};\\
            \draw [line width=\aw,color=cyan!90!green] (\init,0) to ++(\al,0) node[right, color=black] {RNS--$128$};\\
            \draw [line width=\aw,color=blue!60!black] (\init,0) to ++(\al,0) node[right, color=black] {RNS--$256$};\\
            \draw [line width=\aw,color=violet!90!black] (\init,0) to ++(\al,0) node[right, color=black] {RNS--$512$};\\

            \draw [line width=\aw] (\init,0) to ++(\al,0) node[right, color=black] {GAP};\\

            \draw [line width=\aw,color=lime!50!black] (\init,0) to ++(\al,0) node[right, color=black] {$2\times 2$, trans conv, $/2$};\\

            \draw [line width=\aw,color=red!40!white] (\init,0) to ++(\al,0) node[right, color=black] {Bilinear upscale};\\

            \draw [line width=\aw,color=red] (\init,0) to ++(\al,0) node[right, color=black] {$1\times 1$ conv};\\

            \draw [line width=\aw,color=magenta] (\init,0) to ++(\al,0) node[right, color=black] {Fully connected};\\
            \draw [-{Triangle[scale=.5]},line width=.3mm,color=black,dashed] (\init,0) to ++(\al,0) node[right, color=black] {Identity};\\

            \draw [-{Triangle[scale=.5]},line width=\aw,color=black] (\init,0) to ++(\al,0) node[right, color=black] {ReLU};
            \draw [-{Arc Barb[scale=.5]},line width=\aw,color=black] (-\init,0) to ++(\al,0) node[right, color=black] {Sigmoid};\\
            \draw [line width=\aw,color=black] (\init,0) to ++(\al,0) node[right, color=black] {None};
            \draw [-{Bar[scale=.5]},line width=\aw,color=black] (-\init,0) to ++(\al,0) node[right, color=black] {Softmax};\\
            \node (mod1) at (0,.25) [draw,minimum width=\init/2 cm,minimum height=.2 cm, fill=none] {\footnotesize MOD1} node[right, color=black] {$3 \times 3$ avgpool, $/2$, $+1$};\\
            \node (mod2) at (0,.25) [draw,minimum width=\init/2 cm,minimum height=.2 cm, fill=none] {\footnotesize MOD2} node[right, color=black] {$2 \times 2$ avgpool, $/2$, $+1$,};\\
            \node (mod22) at (0,.25) [draw=none,minimum width=\init/2 cm,minimum height=.2 cm, fill=none] {} node[right, color=black] {$1 \times 1$ conv, BN};\\
        };
    \end{tikzpicture}
\end{subfigure}

%% References
\bibliographystyle{siam}
\bibliography{bibliography}

\end{document}
