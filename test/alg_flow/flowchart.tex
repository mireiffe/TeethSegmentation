\documentclass[10pt]{article}
\usepackage[utf8]{inputenc}
\usepackage{geometry} % to change the page dimensions
\geometry{a4paper}
\usepackage{authblk}

%% Packages
\usepackage{graphicx, subcaption}
\usepackage{amsmath,amssymb,amsthm,amsfonts}
\usepackage{bm}
\usepackage{algorithm, algorithmic}
\usepackage{multirow}
\usepackage{bbm}
\usepackage{hyperref,url}
\usepackage{kotex}
\usepackage[bottom]{footmisc}
\usepackage[labelformat=simple]{subcaption}

\usepackage{tikz}
\usetikzlibrary{shapes, arrows.meta, positioning, calc, arrows, backgrounds}

\captionsetup[figure]{labelsep=period}
% \renewcommand{\thefigure}{\Roman{figure}}
% \captionsetup[subfigure]{labelformat=parens} % default is 'parens'
% \renewcommand{\thesubfigure}{\thefigure.\alph{subfigure}.}
\renewcommand\thesubfigure{(\alph{subfigure})}

%% Theorems
\theoremstyle{plain}
\newtheorem{theorem}{Theorem}[section]
\newtheorem{proposition}[theorem]{Proposition}
\newtheorem{lemma}[theorem]{Lemma}

\theoremstyle{remark}
\newtheorem{remark}[theorem]{Remark}
\newtheorem{example}[theorem]{Example}

\theoremstyle{definition}
\newtheorem{definition}[theorem]{Definition}

%% Equations
\numberwithin{equation}{section}

%% Algorithms
\renewcommand\algorithmicdo{}
\renewcommand\algorithmicthen{}
\renewcommand\algorithmicendfor{\textbf{end}}

%% Macros
\def\chfun{\mathbbm{1}}
\def\calc{\mathcal{C}}
\def\endske{E\left(\mathcal{S}^8\right)}
\def\endomc{E\left(\Omega_c\right)}


\def\gphi{\nabla\phi}
\def\gphii{\nabla\phi_i}
\def\ngphi{\left|\nabla\phi\right|}
\def\ngphii{\left|\nabla\phi_i\right|}

\def\fci{F_{c,\,i}}
\def\fcj{F_{c,\,j}}

\def\fsi{F_{s,\,i}}

\def\cm{\, ,}
\def\pd{\, .}

\def\omi{\Omega_i}
\def\oma{\Omega_a}
\def\ome{\Omega_E}
\def\oms{\Omega_S}
\def\omc{\overline{\Omega}_P}
\def\omcc{\Omega_P}
\def\omt{\Omega_T}
\def\omn{\Omega_N}

\def\ER{\mathcal{R}}
\def\PER{\mathcal{P}}
\def\ERC{\Omega \setminus \ER}

\def\opp{\Omega_{\phi_i}^+}
\def\opn{\Omega_{\phi_i}^-}
% \def\opo{\Omega_{\phi}^0}
\def\opo{C_{\phi_i}}

\def\sdf{\mbox{\textrm{sdf}}}

\DeclareMathOperator*{\argmax}{arg\,max}
\DeclareMathOperator*{\argmin}{arg\,min}
% \def\tOmega{\tilde{\Omega}}
% \def\tGamma{\tilde{\Gamma}}
% \def\tV{\tilde{V}}
% \def\W{\mathbb{W}}
% \def\tu{\tilde{u}}
% \def\bu{\bar{u}}
% \def\hu{\hat{u}}
% \def\bl{\bar{\lambda}}
% \def\p{\mathbf{p}}
% \def\P{\mathbf{P}}
% \def\intO{\int_{\Omega}}
% \def\intOs{\int_{\Omega_s}}
% \def\m{\mathbf{m}}
% \def\n{\mathbf{n}}
% \def\blambda{\bm{\lambda}}

% \def\tE{\tilde{E}}
% \def\N{\mathcal{N}}

% \def\div{\mathrm{div}}
% \def\proj{\mathrm{proj}}
% \def\prox{\mathrm{prox}}
% \def\ran{\mathrm{ran}\,}
% \def\ed{\mathrm{ed}}
% \def\supp{\mathrm{supp}\,}
% \def\TOL{\mathrm{TOL}}
% \DeclareMathOperator*{\argmin}{\arg\min}

% Text Color and Strike
\usepackage[normalem]{ulem}
\usepackage{color}
\newcommand{\red}[1]{{\color{red}{#1}}}
\newcommand{\blue}[1]{{\color{blue}{#1}}}
\newcommand{\blfootnote}[1]{%
  \begingroup
  \renewcommand\thefootnote{}\footnote{#1}%
  \addtocounter{footnote}{-1}%
  \endgroup
}

\title{ Individual Tooth Segmentation in Human Teeth Images Using Pseudo Edge-Regions Obtained by Deep Neural Networks }
\author{Seongeun Kim and Chang-Ock Lee}
\affil{Department of Mathematical Sciences, KAIST, Daejeon 34141, Korea}
\date{ }

\begin{document}
\maketitle

\begin{figure}
	\newcommand*{\algvgap}{1.8}%
	\newcommand*{\segvgap}{0.45}%
	\newcommand*{\midgap}{1.4}%
	\newcommand*{\algminwid}{2.8}%
	\newcommand*{\algminhei}{1.1}%
	\newcommand*{\minwid}{2}%
	\newcommand*{\minhei}{1.7}%
	\centering
	\begin{tikzpicture}[font=\normalsize, thick, node distance=1.5cm and 1.5cm, scale=0.8, every node/.style={scale=0.75}]
	\tikzstyle{inter} = [
		draw,
		trapezium, 
		trapezium left angle = 65,
		trapezium right angle = 115,
		trapezium stretches,
		minimum width=\minwid cm,
		minimum height=\minhei cm,
		align=center,
		text width=0.7*\minwid cm,
		fill=blue!20
		]
	\tikzstyle{io} = [
		draw,
		rounded rectangle,
		minimum width=.8*\minwid cm,
		minimum height=\minhei cm,
		text width=.8*\minwid cm,
		align=center,
		fill=red!30
		]
	\tikzstyle{proc} = [
		draw,
		minimum width=\minwid cm,
		minimum height=\minhei cm,
		text width=\minwid cm,
		align=center,
	    fill=orange!20,
		]
	\tikzstyle{alginter} = [
		draw,
		trapezium, 
		trapezium left angle = 65,
		trapezium right angle = 115,
		trapezium stretches,
		minimum width=\algminwid cm,
		minimum height=\algminhei cm,
		align=center,
		text width=0.7*\algminwid cm,
		fill=blue!20
		]
	\tikzstyle{algproc} = [
		draw,
		minimum width=\algminwid cm,
		minimum height=\algminhei cm,
		text width=\algminwid cm,
		align=center,
	    fill=orange!20,
		]
		
	%segmentation
	\node [io,
		% below=0.8*\segvgap cm of title_b,
		] (input) at (-3, -4.5) {Teeth\\image};

	\node [inter,
		right=\segvgap cm of input,
		] (dnn) {Deep\\neural\\network};

	\node [inter,
		right=\segvgap cm of dnn,
		] (per) {Pseudo edge-region};

	\node [proc,
		right=\segvgap cm of per,
		] (ac) {Active contour models};

	\node [inter,
		right=\segvgap cm of ac,
		] (regions) {Segmented regions};
		
	\node [proc,
		right=\segvgap cm of regions
		] (id) {Region\\identification};
		
	\node [io,
		right=\segvgap cm of id,
		] (res) {Result};
		
	% labeling
	\node [alginter,
	% above=2.3*\midgap cm of input,
		] (image) at (-3 + .5*\algminwid - .5*\minwid-.05, 0) { Teeth image };
	
	\node [algproc,
		above=0.75*\algvgap cm of dnn,
		] (train) {Training};

    \node [algproc,
		right=\algvgap cm of image,
		] (ome) {$\ome$ from GADF};

	\node [algproc,
		right=\algvgap cm of ome,
		] (selection) {{Manual selection} \\ {\& dilation}};

	\node [alginter,
		right=\algvgap cm of selection,
		] (label) {Label};

	% Arrows
	\draw[thick,-latex] (image) edge (ome)
		(ome) edge (selection)
		(selection) edge (label)
		(train) edge (dnn);

	\draw[thick,-latex] (input) edge (dnn)
		(dnn) edge (per)
		(per) edge (ac)
		(ac) edge (regions)
		(regions) edge (id)
		(id) edge (res);

	\draw[thick,-latex] (image) |- (train);
	\draw[thick,-latex] (label) |- (train);
	
	\node [draw,
		minimum width=3cm,
		minimum height=1.0cm,
		fill=white,
		above=.20*\midgap cm of ac
		] (title_b) {\textbf{(b) Segmentation}};
	
	\node [draw,
		minimum width=5cm,
		minimum height=1cm,
		above=1.85*\midgap cm of title_b,
		fill=white
		] (title_a) {\textbf{(a) Data labeling and training}};
	
	\begin{scope}[on background layer]
		\draw[black,line width=.4mm,dashed] ($(title_a.west)+(-6.77,0)$)  rectangle ($(title_a.west)+(-6.77,0)+(18.2,-8*\segvgap)$);
		\draw[black,line width=.4mm,dashed] ($(title_b.west)+(-7.7,0)$)  rectangle ($(title_b.west)+(-7.7,0)+(18.2,-6.3*\segvgap)$);
	\end{scope}
  \end{tikzpicture}
  \caption{Algorithm flowchart. }
  \label{Fig:flowchart}
\end{figure}
\end{document}