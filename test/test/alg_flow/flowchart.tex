%%
%% This is file `sample-manuscript.tex',
%% generated with the docstrip utility.
%%
%% The original source files were:
%%
%% samples.dtx  (with options: `manuscript')
%% 
%% IMPORTANT NOTICE:
%% 
%% For the copyright see the source file.
%% 
%% Any modified versions of this file must be renamed
%% with new filenames distinct from sample-manuscript.tex.
%% 
%% For distribution of the original source see the terms
%% for copying and modification in the file samples.dtx.
%% 
%% This generated file may be distributed as long as the
%% original source files, as listed above, are part of the
%% same distribution. (The sources need not necessarily be
%% in the same archive or directory.)
%%
%% Commands for TeXCount
%TC:macro \cite [option:text,text]
%TC:macro \citep [option:text,text]
%TC:macro \citet [option:text,text]
%TC:envir table 0 1
%TC:envir table* 0 1
%TC:envir tabular [ignore] word
%TC:envir displaymath 0 word
%TC:envir math 0 word
%TC:envir comment 0 0
%%
%%
%% The first command in your LaTeX source must be the \documentclass command.
% \documentclass[manuscript,screen,review]{acmart}
\documentclass[acmtog,anonymous,review]{acmart}

% \usepackage{subfigure}
\usepackage{subcaption}
\usepackage{tikz}
\usetikzlibrary{shapes, arrows.meta, positioning, calc, arrows, backgrounds}


%macros
\def\calc{\mathcal{C}}

\def\gphi{\nabla\phi}
\def\gphii{\nabla\phi_i}
\def\ngphi{\left|\nabla\phi\right|}
\def\ngphii{\left|\nabla\phi_i\right|}

\def\fci{F_{c,i}}
\def\fcj{F_{c,j}}
\def\fsi{F_{s,i}}

\def\cm{\, ,}
\def\pd{\, .}

\def\oma{\Omega_a}
\def\ome{\Omega_E}

\def\ER{\mathcal{R}}
\def\PER{\mathcal{P}}
\def\ERC{\Omega \setminus \ER}

\def\sdf{\mbox{\textrm{sdf}}}

%%
%% \BibTeX command to typeset BibTeX logo in the docs
\AtBeginDocument{%
  \providecommand\BibTeX{{%
    \normalfont B\kern-0.5em{\scshape i\kern-0.25em b}\kern-0.8em\TeX}}}

%% Rights management information.  This information is sent to you
%% when you complete the rights form.  These commands have SAMPLE
%% values in them; it is your responsibility as an author to replace
%% the commands and values with those provided to you when you
%% complete the rights form.
\setcopyright{acmcopyright}
\copyrightyear{2022}
\acmYear{2022}
\acmDOI{XXXXXXX.XXXXXXX}

%% These commands are for a PROCEEDINGS abstract or paper.
% \acmConference[Conference acronym 'XX]{Make sure to enter the correct
%   conference title from your rights confirmation emai}{June 03--05,
%   2018}{Woodstock, NY}
% \acmPrice{15.00}
% \acmISBN{978-1-4503-XXXX-X/18/06}


%%
%% Submission ID.
%% Use this when submitting an article to a sponsored event. You'll
%% receive a unique submission ID from the organizers
%% of the event, and this ID should be used as the parameter to this command.
%%\acmSubmissionID{123-A56-BU3}

%%
%% For managing citations, it is recommended to use bibliography
%% files in BibTeX format.
%%
%% You can then either use BibTeX with the ACM-Reference-Format style,
%% or BibLaTeX with the acmnumeric or acmauthoryear sytles, that include
%% support for advanced citation of software artefact from the
%% biblatex-software package, also separately available on CTAN.
%%
%% Look at the sample-*-biblatex.tex files for templates showcasing
%% the biblatex styles.
%%

%%
%% The majority of ACM publications use numbered citations and
%% references.  The command \citestyle{authoryear} switches to the
%% "author year" style.
%%
%% If you are preparing content for an event
%% sponsored by ACM SIGGRAPH, you must use the "author year" style of
%% citations and references.
%% Uncommenting
%% the next command will enable that style.
\citestyle{acmauthoryear}

%%
%% end of the preamble, start of the body of the document source.
\begin{document}

%%
%% The "title" command has an optional parameter,
%% allowing the author to define a "short title" to be used in page headers.
% \title{Individual Tooth Segmentation in Human Teeth Images Using Pseudo Edge-Region Obtained by Deep Neural Networks}
\title[Individual Tooth Segmentation in Human Teeth Images]{Individual Tooth Segmentation in Human Teeth Images Using Pseudo Edge-Region Obtained by Deep Neural Networks}

%%
%% The "author" command and its associated commands are used to define
%% the authors and their affiliations.
%% Of note is the shared affiliation of the first two authors, and the
%% "authornote" and "authornotemark" commands
%% used to denote shared contribution to the research.
\author{Seongeun Kim}
\authornote{Both authors contributed equally to this research.}
\email{mireiffe@kaist.ac.kr}
\orcid{0000-0002-2778-5042}

\author{Chang-Ock Lee}
\authornotemark[1]
\email{colee@kaist.edu}
\affiliation{%
  \institution{Korea Advanced Institute of Science and Technology}
  \streetaddress{291 Daehak-ro, Yuseong-gu}
  \city{Daejeon}
  % \state{}
  \country{Republic of Korea}
  \postcode{34141}
}

%%
%% By default, the full list of authors will be used in the page
%% headers. Often, this list is too long, and will overlap
%% other information printed in the page headers. This command allows
%% the author to define a more concise list
%% of authors' names for this purpose.
% \renewcommand{\shortauthors}{Trovato and Tobin, et al.}

%%
%% The abstract is a short summary of the work to be presented in the
%% article.
\begin{abstract}
\end{abstract}

%%
%% The code below is generated by the tool at http://dl.acm.org/ccs.cfm.
%% Please copy and paste the code instead of the example below.
%%
\begin{CCSXML}
  <ccs2012>
    <concept>
        <concept_id>10010147.10010178.10010224.10010245.10010247</concept_id>
        <concept_desc>Computing methodologies~Image segmentation</concept_desc>
        <concept_significance>500</concept_significance>
        </concept>
    <concept>
        <concept_id>10010147.10010257.10010293.10010294</concept_id>
        <concept_desc>Computing methodologies~Neural networks</concept_desc>
        <concept_significance>500</concept_significance>
        </concept>
  </ccs2012>
\end{CCSXML}

\ccsdesc[500]{Computing methodologies~Image segmentation}
\ccsdesc[500]{Computing methodologies~Neural networks}

%%
%% Keywords. The author(s) should pick words that accurately describe
%% the work being presented. Separate the keywords with commas.
\keywords{Tooth segmentation, Neural network, Geometric attraction-driven flow, Edge-region, Light reflection}


%%
%% This command processes the author and affiliation and title
%% information and builds the first part of the formatted document.
\maketitle
 
% Section: Introduction
\section{Introduction}
\label{Sec:Introduction}

\begin{figure*}
	\newcommand*{\algvgap}{1.8}%
	\newcommand*{\segvgap}{0.45}%
	\newcommand*{\midgap}{1.4}%
	\newcommand*{\algminwid}{2.8}%
	\newcommand*{\algminhei}{1.1}%
	\newcommand*{\minwid}{2}%
	\newcommand*{\minhei}{1.7}%
	\centering
	\begin{tikzpicture}[font=\normalsize, thick, node distance=1.5cm and 1.5cm, scale=0.8, every node/.style={scale=0.75}]
	\tikzstyle{inter} = [
		draw,
		trapezium, 
		trapezium left angle = 65,
		trapezium right angle = 115,
		trapezium stretches,
		minimum width=\minwid cm,
		minimum height=\minhei cm,
		align=center,
		text width=0.7*\minwid cm,
		fill=blue!20
		]
	\tikzstyle{io} = [
		draw,
		rounded rectangle,
		minimum width=.8*\minwid cm,
		minimum height=\minhei cm,
		text width=.8*\minwid cm,
		align=center,
		fill=red!30
		]
	\tikzstyle{proc} = [
		draw,
		minimum width=\minwid cm,
		minimum height=\minhei cm,
		text width=\minwid cm,
		align=center,
	    fill=orange!20,
		]
	\tikzstyle{alginter} = [
		draw,
		trapezium, 
		trapezium left angle = 65,
		trapezium right angle = 115,
		trapezium stretches,
		minimum width=\algminwid cm,
		minimum height=\algminhei cm,
		align=center,
		text width=0.7*\algminwid cm,
		fill=blue!20
		]
	\tikzstyle{algproc} = [
		draw,
		minimum width=\algminwid cm,
		minimum height=\algminhei cm,
		text width=\algminwid cm,
		align=center,
	    fill=orange!20,
		]
		
	%segmentation
	\node [io,
		% below=0.8*\segvgap cm of title_b,
		] (input) at (-3, -4.5) {Teeth\\image};

	\node [inter,
		right=\segvgap cm of input,
		] (dnn) {Deep\\neural\\network};

	\node [inter,
		right=\segvgap cm of dnn,
		] (per) {Pseudo edge-region};

	\node [proc,
		right=\segvgap cm of per,
		] (ac) {Active contour models};

	\node [inter,
		right=\segvgap cm of ac,
		] (regions) {Segmented regions};
		
	\node [proc,
		right=\segvgap cm of regions
		] (id) {Region\\identification};
		
	\node [io,
		right=\segvgap cm of id,
		] (res) {Result};
		
	% labeling
	\node [alginter,
	% above=2.3*\midgap cm of input,
		] (image) at (-3 + .5*\algminwid - .5*\minwid-.05, 0) { Teeth image };
	
	\node [algproc,
		above=0.75*\algvgap cm of dnn,
		] (train) {Training};

    \node [algproc,
		right=\algvgap cm of image,
		] (ome) {$\ome$ from GADF};

	\node [algproc,
		right=\algvgap cm of ome,
		] (selection) {{Manual selection} \\ {\& dilation}};

	\node [alginter,
		right=\algvgap cm of selection,
		] (label) {Label};

	% Arrows
	\draw[thick,-latex] (image) edge (ome)
		(ome) edge (selection)
		(selection) edge (label)
		(train) edge (dnn);

	\draw[thick,-latex] (input) edge (dnn)
		(dnn) edge (per)
		(per) edge (ac)
		(ac) edge (regions)
		(regions) edge (id)
		(id) edge (res);

	\draw[thick,-latex] (image) |- (train);
	\draw[thick,-latex] (label) |- (train);
	
	\node [draw,
		minimum width=3cm,
		minimum height=1.0cm,
		fill=white,
		above=.20*\midgap cm of ac
		] (title_b) {\textbf{(b) Segmentation}};
	
	\node [draw,
		minimum width=5cm,
		minimum height=1cm,
		above=1.85*\midgap cm of title_b,
		fill=white
		] (title_a) {\textbf{(a) Data labeling and training}};
	
	\begin{scope}[on background layer]
		\draw[black,line width=.4mm,dashed] ($(title_a.west)+(-6.77,0)$)  rectangle ($(title_a.west)+(-6.77,0)+(18.2,-8*\segvgap)$);
		\draw[black,line width=.4mm,dashed] ($(title_b.west)+(-7.7,0)$)  rectangle ($(title_b.west)+(-7.7,0)+(18.2,-6.3*\segvgap)$);
	\end{scope}
  \end{tikzpicture}
  \caption{Algorithm flowchart. }
  \label{Fig:flowchart}
\end{figure*}

\end{document}