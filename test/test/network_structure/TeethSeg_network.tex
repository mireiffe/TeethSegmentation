%%
%% This is file `sample-manuscript.tex',
%% generated with the docstrip utility.
%%
%% The original source files were:
%%
%% samples.dtx  (with options: `manuscript')
%% 
%% IMPORTANT NOTICE:
%% 
%% For the copyright see the source file.
%% 
%% Any modified versions of this file must be renamed
%% with new filenames distinct from sample-manuscript.tex.
%% 
%% For distribution of the original source see the terms
%% for copying and modification in the file samples.dtx.
%% 
%% This generated file may be distributed as long as the
%% original source files, as listed above, are part of the
%% same distribution. (The sources need not necessarily be
%% in the same archive or directory.)
%%
%% Commands for TeXCount
%TC:macro \cite [option:text,text]
%TC:macro \citep [option:text,text]
%TC:macro \citet [option:text,text]
%TC:envir table 0 1
%TC:envir table* 0 1
%TC:envir tabular [ignore] word
%TC:envir displaymath 0 word
%TC:envir math 0 word
%TC:envir comment 0 0
%%
%%
%% The first command in your LaTeX source must be the \documentclass command.
% \documentclass[manuscript,screen,review]{acmart}
\documentclass[acmtog,anonymous,review]{acmart}

% \usepackage{subfigure}
\usepackage{subcaption}
\usepackage{tikz}
\usetikzlibrary{shapes, arrows.meta, positioning, calc, arrows, backgrounds}


%macros
\def\calc{\mathcal{C}}

\def\gphi{\nabla\phi}
\def\gphii{\nabla\phi_i}
\def\ngphi{\left|\nabla\phi\right|}
\def\ngphii{\left|\nabla\phi_i\right|}

\def\fci{F_{c,i}}
\def\fcj{F_{c,j}}
\def\fsi{F_{s,i}}

\def\cm{\, ,}
\def\pd{\, .}

\def\oma{\Omega_a}
\def\ome{\Omega_E}

\def\ER{\mathcal{R}}
\def\PER{\mathcal{P}}
\def\ERC{\Omega \setminus \ER}

\def\sdf{\mbox{\textrm{sdf}}}

%%
%% \BibTeX command to typeset BibTeX logo in the docs
\AtBeginDocument{%
  \providecommand\BibTeX{{%
    \normalfont B\kern-0.5em{\scshape i\kern-0.25em b}\kern-0.8em\TeX}}}

%% Rights management information.  This information is sent to you
%% when you complete the rights form.  These commands have SAMPLE
%% values in them; it is your responsibility as an author to replace
%% the commands and values with those provided to you when you
%% complete the rights form.
\setcopyright{acmcopyright}
\copyrightyear{2022}
\acmYear{2022}
\acmDOI{XXXXXXX.XXXXXXX}

%% These commands are for a PROCEEDINGS abstract or paper.
% \acmConference[Conference acronym 'XX]{Make sure to enter the correct
%   conference title from your rights confirmation emai}{June 03--05,
%   2018}{Woodstock, NY}
% \acmPrice{15.00}
% \acmISBN{978-1-4503-XXXX-X/18/06}


%%
%% Submission ID.
%% Use this when submitting an article to a sponsored event. You'll
%% receive a unique submission ID from the organizers
%% of the event, and this ID should be used as the parameter to this command.
%%\acmSubmissionID{123-A56-BU3}

%%
%% For managing citations, it is recommended to use bibliography
%% files in BibTeX format.
%%
%% You can then either use BibTeX with the ACM-Reference-Format style,
%% or BibLaTeX with the acmnumeric or acmauthoryear sytles, that include
%% support for advanced citation of software artefact from the
%% biblatex-software package, also separately available on CTAN.
%%
%% Look at the sample-*-biblatex.tex files for templates showcasing
%% the biblatex styles.
%%

%%
%% The majority of ACM publications use numbered citations and
%% references.  The command \citestyle{authoryear} switches to the
%% "author year" style.
%%
%% If you are preparing content for an event
%% sponsored by ACM SIGGRAPH, you must use the "author year" style of
%% citations and references.
%% Uncommenting
%% the next command will enable that style.
\citestyle{acmauthoryear}

%%
%% end of the preamble, start of the body of the document source.
\begin{document}

%%
%% The "title" command has an optional parameter,
%% allowing the author to define a "short title" to be used in page headers.
% \title{Individual Tooth Segmentation in Human Teeth Images Using Pseudo Edge-Region Obtained by Deep Neural Networks}
\title[Individual Tooth Segmentation in Human Teeth Images]{Individual Tooth Segmentation in Human Teeth Images Using Pseudo Edge-Region Obtained by Deep Neural Networks}

%%
%% The "author" command and its associated commands are used to define
%% the authors and their affiliations.
%% Of note is the shared affiliation of the first two authors, and the
%% "authornote" and "authornotemark" commands
%% used to denote shared contribution to the research.
\author{Seongeun Kim}
\authornote{Both authors contributed equally to this research.}
\email{mireiffe@kaist.ac.kr}
\orcid{0000-0002-2778-5042}

\author{Chang-Ock Lee}
\authornotemark[1]
\email{colee@kaist.edu}
\affiliation{%
  \institution{Korea Advanced Institute of Science and Technology}
  \streetaddress{291 Daehak-ro, Yuseong-gu}
  \city{Daejeon}
  % \state{}
  \country{Republic of Korea}
  \postcode{34141}
}

%%
%% By default, the full list of authors will be used in the page
%% headers. Often, this list is too long, and will overlap
%% other information printed in the page headers. This command allows
%% the author to define a more concise list
%% of authors' names for this purpose.
% \renewcommand{\shortauthors}{Trovato and Tobin, et al.}

%%
%% The abstract is a short summary of the work to be presented in the
%% article.
\begin{abstract}
\end{abstract}

%%
%% The code below is generated by the tool at http://dl.acm.org/ccs.cfm.
%% Please copy and paste the code instead of the example below.
%%
\begin{CCSXML}
  <ccs2012>
    <concept>
        <concept_id>10010147.10010178.10010224.10010245.10010247</concept_id>
        <concept_desc>Computing methodologies~Image segmentation</concept_desc>
        <concept_significance>500</concept_significance>
        </concept>
    <concept>
        <concept_id>10010147.10010257.10010293.10010294</concept_id>
        <concept_desc>Computing methodologies~Neural networks</concept_desc>
        <concept_significance>500</concept_significance>
        </concept>
  </ccs2012>
\end{CCSXML}

\ccsdesc[500]{Computing methodologies~Image segmentation}
\ccsdesc[500]{Computing methodologies~Neural networks}

%%
%% Keywords. The author(s) should pick words that accurately describe
%% the work being presented. Separate the keywords with commas.
\keywords{Tooth segmentation, Neural network, Geometric attraction-driven flow, Edge-region, Light reflection}


%%
%% This command processes the author and affiliation and title
%% information and builds the first part of the formatted document.
\maketitle
 
% Section: Introduction
\section{Introduction}
\label{Sec:Introduction}

\begin{figure*}[]
    \centering
    \begin{subfigure}[b]{1\textwidth}
        \newcommand*{\al}{0.3}%
        \newcommand*{\aw}{1.mm}%
        \newcommand*{\hei}{1.1 cm}%
        \newcommand*{\flr}{3.5cm}%
        \newcommand*{\wgap}{1cm}%
        \newcommand*{\init}{0}%
        \newcommand*{\wwa}{.2}%
        \newcommand*{\wwb}{.3}%
        \newcommand*{\wwc}{.4}%
        \newcommand*{\wwd}{.5}%
        \newcommand*{\wwe}{.6}%
        \newcommand*{\wwf}{.7}%
        \newcommand*{\wwg}{.9}%
    
        \newcommand*{\hha}{1}%
        \newcommand*{\hhb}{.9}%
        \newcommand*{\hhc}{.8}%
        \newcommand*{\hhd}{.6}%
        \newcommand*{\hhe}{.4}%
        \newcommand*{\hhf}{.3}%
        \newcommand*{\hhg}{.2}%
        \centering
        \begin{tikzpicture}[]
            \footnotesize
            \node (x0) at (0,0) [draw=none,minimum width=1mm,minimum height=\hei, fill=blue!50!white,label=above:{$3$}] {};
            \draw [-{Triangle[scale=.5]},line width=\aw,color=red!50!yellow] (x0.south) to ++(0,-\al);
    
            \node (x1) at ($(x0) + (0,-\hei-\al cm - .3 cm)$) [draw=none,minimum width=\wwa cm,minimum height=\hhb*\hei, fill=blue!50!white,label=above:{$32$}] {}; 
            \draw [-{Triangle[scale=.5]},line width=\aw,color=green!80!black] (x1.east) -- ++(\al, 0);
    
            \node (x2) at ($(x1) + (\al+ \wwa/2 + \wwa/2,0)$) [draw=none,minimum width=\wwa cm,minimum height=\hhb*\hei, fill=blue!50!white,label=above:{$32$}] {};
            \draw [-{Triangle[scale=.5]},line width=\aw,color=green!80!black] (x2.east) -- ++(\al, 0);
            
            \node (x3) at ($(x2) + (\al+\wwa/2 + \wwb/2,0)$) [draw=none,minimum width=\wwb cm,minimum height=\hhb*\hei, fill=blue!50!white,label=above:{$64$}] {};
            \draw [-{Triangle[scale=.5]},line width=\aw,color=black] (x3.south) to ++(0,-\al);
            
            \node (x4) at ($(x3) + (0,-\hhb*\hei-\al cm - .3 cm)$) [draw=none,minimum width=\wwb cm,minimum height=\hhc*\hei, fill=blue!50!white,label=above:{$64$}] {};
            \draw [-{Triangle[scale=.5]},line width=\aw,color=cyan!90!white] (x4.east) to node [above, color=black] {} ++(\al, 0);
            
            \node (x5) at ($(x4) + (\al+\wwb/2 + \wwd/2,0)$) [draw=none,minimum width=\wwd cm,minimum height=\hhc*\hei, fill=blue!50!white,label=above:{$256$}] {};
            \draw [-{Triangle[scale=.5]},line width=\aw,color=cyan!90!white] (x5.east) -- node [above, color=black] {\scriptsize $\boxed{2}$} ++(2*\al, 0);
    
            \node (x6) at ($(x5) + (2*\al+\wwd/2 + \wwd/2,0)$) [draw=none,minimum width=\wwd cm,minimum height=\hhc*\hei, fill=blue!50!white,label=above:{$256$}] {};
            \draw [-{Triangle[scale=.5]},line width=\aw,color=cyan!90!white] (x6.south) to  ++(0,-\al) {};
    
            \node (x7) at ($(x6) + (0,-\hhc*\hei-\al cm - .3 cm)$) [draw=none,minimum width=\wwe cm,minimum height=\hhd*\hei, fill=blue!50!white,label=above:{$512$}] {};
            \draw [-{Triangle[scale=.5]},line width=\aw,color=cyan!90!white] (x7.east) to node [above, color=black] {\scriptsize $\boxed{3}$} ++(2*\al,0) {};
    
            \node (x8) at ($(x7) + (2*\al+\wwe/2 + \wwe/2,0)$) [draw=none,minimum width=\wwe cm,minimum height=\hhd*\hei, fill=blue!50!white,label=above:{$512$}] {};
            \draw [-{Triangle[scale=.5]},line width=\aw,color=cyan!90!white] (x8.south) to  ++(0,-\al) {};
    
            \node (x9) at ($(x8) + (0,-\hhd*\hei-\al cm - .3 cm)$) [draw=none,minimum width=\wwf cm,minimum height=\hhe*\hei, fill=blue!50!white,label=above:{$1024$}] {};
            \draw [-{Triangle[scale=.5]},line width=\aw,color=cyan!90!white] (x9.east) to node [above,color=black] {\scriptsize $\boxed{5}$} ++(2*\al,0) {};
    
            \node (x10) at ($(x9) + (2*\al+\wwf/2 + \wwf/2,0)$) [draw=none,minimum width=\wwf cm,minimum height=\hhe*\hei, fill=blue!50!white,label=above:{$1024$}] {};
            \draw [-{Triangle[scale=.5]},line width=\aw,color=cyan!90!white] (x10.south) to  ++(0,-\al) {};
    
            \node (x11) at ($(x10) + (0,-\hhe*\hei-\al cm - .3 cm)$) [draw=none,minimum width=\wwg cm,minimum height=\hhf*\hei, fill=blue!50!white,label=above:{$2048$}] {};
            \draw [-{Triangle[scale=.5]},line width=\aw,color=cyan!90!white] (x11.east) to node [above, color=black] {\scriptsize $\boxed{2}$} ++(3*\al,0) {};
            
            \node (x12) at ($(x11) + (3*\al+\wwg/2 + \wwg/2,0)$) [draw=none,minimum width=\wwg cm,minimum height=\hhf*\hei, fill=blue!50!white,label=above:{$2048$}] {};
            
            % latent variable
            % \node (x13) at ($(x12) + (1.5*\al+\wwg/2 + \wwg/2,0)$) [draw=none,minimum width=\wwg cm,minimum height=2mm, fill=blue!50!white,label=above:{$2048$}] {};
            % \draw [line width=\aw] (x12.east) to (x13.west) {};
            
            \node (x14) at ($(x12) + (0,\hhe*\hei+\al cm + .3 cm)$) [draw=none,minimum width=\wwg cm,minimum height=\hhe*\hei, fill=blue!50!white,label=above:{$2048$}] {};
            \draw [{Triangle[scale=.5]}-,line width=\aw,color=yellow!90!black] (x14.south) to ++(0,-\al);
            
            \node (x15) at ($(x14) + (1.5*\al + \wwg/2 + \wwf/2,0)$) [draw=none,minimum width=\wwf cm,minimum height=\hhe*\hei, fill=blue!50!white,label=above:{$1024$}] {};
            \draw [-{Triangle[scale=.5]},line width=\aw,color=green!80!black] (x14.east) to node [above, color=black] {\scriptsize $\boxed{2}$} (x15.west) {};
    
            \node (x16) at ($(x15) + (0,\hhd*\hei+\al cm + .3 cm)$) [draw=none,minimum width=\wwf cm,minimum height=\hhd*\hei, fill=blue!50!white,label=above:{$1024$}] {};
            \draw [{Triangle[scale=.5]}-,line width=\aw,color=yellow!90!black] (x16.south) to ++(0,-\al);
            
            \node (x17) at ($(x16) + (1.5*\al + \wwe/2 + \wwf/2,0)$) [draw=none,minimum width=\wwe cm,minimum height=\hhd*\hei, fill=blue!50!white,label=above:{$512$}] {};
            \draw [-{Triangle[scale=.5]},line width=\aw,color=green!80!black] (x16.east) to node [above, color=black] {\scriptsize $\boxed{2}$} (x17.west) {};
    
            \node (x18) at ($(x17) + (0,\hhc*\hei+\al cm + .3 cm)$) [draw=none,minimum width=\wwe cm,minimum height=\hhc*\hei, fill=blue!50!white,label=above:{$512$}] {};
            \draw [{Triangle[scale=.5]}-,line width=\aw,color=yellow!90!black] (x18.south) to ++(0,-\al);
            
            \node (x19) at ($(x18) + (1.5*\al + \wwd/2 + \wwe/2,0)$) [draw=none,minimum width=\wwd cm,minimum height=\hhc*\hei, fill=blue!50!white,label=above:{$256$}] {};
            \draw [-{Triangle[scale=.5]},line width=\aw,color=green!80!black] (x18.east) to node [above, color=black] {\scriptsize $\boxed{2}$} (x19.west) {};
    
            \node (x20) at ($(x19) + (0,\hhb*\hei+\al cm + .3 cm)$) [draw=none,minimum width=\wwd cm,minimum height=\hhb*\hei, fill=blue!50!white,label=above:{$256$}] {};
            \draw [{Triangle[scale=.5]}-,line width=\aw,color=violet!90!white] (x20.south) to ++(0,-\al);
            
            \node (x21) at ($(x20) + (1.5*\al + \wwd/2 + \wwc/2,0)$) [draw=none,minimum width=\wwc cm,minimum height=\hhb*\hei, fill=blue!50!white,label=above:{$128$}] {};
            \draw [-{Triangle[scale=.5]},line width=\aw,color=green!80!black] (x20.east) to node [above, color=black] {\scriptsize $\boxed{2}$} (x21.west) {};
    
            \node (x22) at ($(x21) + (0,\hei+\al cm + .3 cm)$) [draw=none,minimum width=\wwc cm,minimum height=\hei, fill=blue!50!white,label=above:{$128$}] {};
            \draw [{Triangle[scale=.5]}-,line width=\aw,color=violet!90!white] (x22.south) to ++(0,-\al);
            
            \node (x23) at ($(x22) + (1.5*\al + \wwb/2 + \wwc/2,0)$) [draw=none,minimum width=\wwb cm,minimum height=\hei, fill=blue!50!white,label=above:{$64$}] {};
            \draw [-{Triangle[scale=.5]},line width=\aw,color=green!80!black] (x22.east) to node [above, color=black] {\scriptsize $\boxed{2}$} (x23.west) {};
            
            \node (x24) at ($(x23) + (\al + \wwb/2 + .1,0)$) [draw=none,minimum width=.1 cm,minimum height=\hei, fill=blue!50!white,label=above:{$1$}] {};
            \draw [-{Circle[scale=.5]},line width=\aw,,color=red] (x23.east) to (x24.west) {};
        \end{tikzpicture}
        \caption{Network structure}
    \end{subfigure}\\
    
    \vspace{-9mm}
    
    \begin{subfigure}[]{.6\textwidth}
        \newcommand*{\hg}{.8}%
        \newcommand*{\wid}{.5}%
        \newcommand*{\hei}{1.3}%
        \newcommand*{\init}{0}%
        \newcommand*{\aw}{1mm}%
        \centering
        \begin{tikzpicture}
            \scriptsize
            \node (ipt) at (0,0) [rotate=90,draw=none,minimum width=\hei cm,minimum height=\wid cm, fill=orange!30!white, label=right:{$c$}, label={[xshift=-.7em, yshift=-.4em, rotate=90]}] {\footnotesize };
            
            \node (x1) at ($(ipt) + (\hg,0)$) [rotate=90,draw=none,minimum width=\hei cm,minimum height=\wid cm, fill=orange!30!white,  label=right:{$m$}, label={[xshift=-.7em, yshift=-.4em, rotate=90]}] {};
    
            \node (x21) at ($(x1) + (\hg,-1.5)$) [rotate=90,draw=none,minimum width=\hei cm,minimum height=\wid cm, fill=orange!30!white, label=right:{$m$}, label={[xshift=-.7em, yshift=-.4em, rotate=90]}] {};
    
            \node (x22) at ($(x1) + (\hg,1.5)$) [rotate=90,draw=none,minimum width=\hei cm,minimum height=\wid cm, fill=orange!30!white, label=right:{$m$}, label={[xshift=-.7em, yshift=-.4em, rotate=90]}] {};
    
            \node (sum1) at ($(x1) + (2*\hg,0)$) [circle, draw, very thick,minimum width=.1 cm,minimum height=.1 cm,text height=.15 cm, fill=none] {\footnotesize $+$};
    
            % \node (x3) at ($(x1) + (3*\hg,0)$) [rotate=90,draw=none,minimum width=\hei cm,minimum height=\wid cm, fill=orange!30!white, label=right:{$m$}, label={[xshift=-.7em, yshift=-.4em, rotate=90]}] {};
    
            \node (x4) at ($(x1) + (3*\hg,0)$) [rotate=90,draw=none,minimum width=.1 cm,minimum height=\wid cm, fill=orange!30!white, label=right:{$m$}, label={[xshift=-.7em, yshift=-.4em, rotate=90]}] {};
    
            \node (x5) at ($(x4) + (\hg,0)$) [rotate=90,draw=none,minimum width=.1 cm,minimum height=.7*\wid cm, fill=orange!30!white, label={[xshift=2.1em, yshift=.8em] $m/2$}, label={[xshift=-.7em, yshift=-.4em, rotate=90]}] {};
    
            \node (x61) at ($(x5) + (0,-.9)$) [rotate=90,draw=none,minimum width=.1 cm,minimum height=\wid cm, fill=orange!30!white, label=right:{$m$}, label={[xshift=-.7em, yshift=-.4em, rotate=90]}] {};
    
            \node (x62) at ($(x5) + (0,.9)$) [rotate=90,draw=none,minimum width=.1 cm,minimum height=\wid cm, fill=orange!30!white, label=right:{$m$}, label={[xshift=-.7em, yshift=-.4em, rotate=90]}] {};
    
            \node (prod11) at ($(x61) + (0,-1.1)$) [circle, draw, very thick,minimum width=.1 cm,minimum height=.1 cm, fill=none] {\footnotesize $\times$};
            \node (prod12) at ($(x62) + (0,1.1)$) [circle, draw, very thick,minimum width=.01 cm,minimum height=.01 cm,fill=none] {\scriptsize $\times$};
    
            \node (sum2) at ($(x5) + (\hg,0)$) [circle, draw, very thick,minimum width=.1 cm,minimum height=.1 cm,text height=.15 cm, fill=none] {\footnotesize $+$};
    
            \node (x7) at ($(x5) + (2*\hg,0)$) [rotate=90,draw=none,minimum width=\hei cm,minimum height=\wid cm, fill=orange!30!white, label=right:{$4m$}, label={[xshift=-.7em, yshift=-.4em, rotate=90]}] {};
    
            \node (sum3) at ($(x7) + (1*\hg,0)$) [circle, draw, very thick,minimum width=.1 cm,minimum height=.1 cm,text height=.15 cm, fill=none] {\footnotesize $+$};
    
            \node (x8) at ($(x7) + (2*\hg,0)$) [rotate=90,draw=none,minimum width=\hei cm,minimum height=\wid cm, fill=orange!30!white, label=right:{$4m$}, label={[xshift=-.7em, yshift=-.4em, rotate=90]}] {};
    
            \draw [-{Triangle[scale=.5]},line width=\aw,color=red] (ipt.south) to (x1.north) {};
            % \draw [-{Triangle[scale=.5]},line width=.2mm,color=black,dashed] (x1.south) to (x1.north) {};
            \draw [-{Triangle[scale=.5]},line width=\aw,color=green!80!black] ($(x1.south) + (0,-.5)$) to (x21.north) {};
            \draw [-{Triangle[scale=.5]},line width=\aw,color=green!80!black] ($(x1.south) + (0,.5)$) to (x22.north) {};
            % \draw [-{Triangle[scale=.5]},line width=\aw] (sum1.east) to (x3.north) {};
            \draw [-{Arc Barb[scale=.5]},line width=\aw] (sum1.east) to (x4.north) {};
            \draw [-{Triangle[scale=.5]},line width=\aw,color=violet!90!white] (x4.south) to (x5.north) {};
            \draw [-{Bar[scale=.5]},line width=.5mm,color=violet!90!white] (x5.west) to ++ (0,-.4);
            \draw [{Bar[scale=.5]}-,line width=.5mm,color=violet!90!white] (x62.west) to ++ (0,-.4);
            \draw [-{Triangle[scale=.5]},line width=\aw,color=red] (sum2.east) to (x7);
            
            \draw [-{Triangle[scale=.5]},line width=.2mm] (x21.south) to (sum1) {};
            \draw [-{Triangle[scale=.5]},line width=.2mm] (x22.south) to (sum1) {};
            \draw [-{Triangle[scale=.5]},line width=.2mm] ($(x21.south) + (0,-.5)$) to (prod11);
            \draw [-{Triangle[scale=.5]},line width=.2mm] ($(x22.south) + (0,.5)$) to (prod12);
            \draw [{Triangle[scale=.5]}-,line width=.2mm] (prod11.north) to (x61.west);
            \draw [{Triangle[scale=.5]}-,line width=.2mm] (prod12.south) to ++(0,-.4);
            \draw [-{Triangle[scale=.5]},line width=.2mm] (prod11.east) to (sum2);
            \draw [-{Triangle[scale=.5]},line width=.2mm] (prod12.east) to (sum2);
            \draw [-{Triangle[scale=.5]},line width=.2mm] (prod12.east) to (sum2);
            \draw [-{Triangle[scale=.5]},line width=.2mm] (x7.south) to (sum3);
            \draw [dashed,line width=.25mm] (x1.west) -- ($(x21.west) + (0,-1cm)$) -- ($(x7.west) + (0,-2.5cm)$) -- (sum3.south) [-{Triangle[scale=.5]}];
            \draw [-{Triangle[scale=.5]},line width=.2mm] (sum3.east) to (x8.north);
        \end{tikzpicture}
        \caption{ResNeSt block with mid-channel size $m$ (RNS-$m$)}
    \end{subfigure}
    \begin{subfigure}[]{.33\textwidth}
        \newcommand*{\hg}{1.2}%
        \newcommand*{\wid}{.5}%
        \newcommand*{\hei}{2}%
        \newcommand*{\init}{-1}%
        \newcommand*{\aw}{.9mm}%
        \newcommand*{\al}{.5}%
        \centering
        \begin{tikzpicture}
            \footnotesize
            \matrix [draw,below left, thick] at ($(current bounding box) - (0, 3)$) {
                \draw [line width=\aw,color=red!50!yellow] (\init,0) to ++(\al,0) node[right, color=black] {$3\times 3$ conv, $/2$, $+1$};\\
    
                \draw [line width=\aw,color=green!80!black] (\init,0) to ++(\al,0) node[right, color=black] {$3\times 3$ conv, $+1$};\\
    
                % \draw [line width=\aw,color=yellow!80!black] (\init,0) to ++(\al,0) node[right, color=black] {$3\times 3$ maxpool, $/2$, $+1$};\\
    
                \draw [line width=\aw] (\init,0) to ++(\al,0) node[right, color=black] {\footnotesize{(a) $3\times 3$ maxpool, $/2$, $+1$}};\\
                \draw [draw=none,minimum width=\init/2 cm,minimum height=.2 cm, fill=none] (\init,0) to ++(\al,0) node[right, color=black] {\,(b) GAP};\\
    
                \draw [line width=\aw,color=cyan!90!white] (\init,0) to ++(\al,0) node[right, color=black] {RNS--$m$};\\
    
                \draw [line width=\aw,color=yellow!90!black] (\init,0) to ++(\al,0) node[right, color=black] {$2\times 2$, trans conv, $/2$};\\
    
                \draw [line width=\aw,color=violet!90!white] (\init,0) to ++(\al,0) node[right, color=black] {(a) Bilinear upscale};\\
                \draw [draw=none,minimum width=\init/2 cm,minimum height=.2 cm, fill=none] (\init,0) to ++(\al,0) node[right, color=black] {\,(b) Fully connected};\\
    
                \draw [line width=\aw,color=red] (\init,0) to ++(\al,0) node[right, color=black] {$1\times 1$ conv};\\
    
                \draw [-{Triangle[scale=.5]},line width=.3mm,color=black] (\init,0) to ++(\al,0) node[right, color=black] {Identity};
                \draw [-{Triangle[scale=.5]},line width=.3mm,color=black,dashed] (-\init,0) to ++(\al,0) node[right, color=black] {Identity};\\
    
                \draw [-{Triangle[scale=.5]},line width=\aw,color=black] (\init,0) to ++(\al,0) node[right, color=black] {ReLU};
                \draw [-{Circle[scale=.5]},line width=\aw,color=black] (-\init,0) to ++(\al,0) node[right, color=black] {Sigmoid};\\
                \draw [-{Arc Barb[scale=.5]},line width=\aw,color=black] (\init,0) to ++(\al,0) node[right, color=black] {None};
                \draw [-{Bar[scale=.5]},line width=\aw,color=black] (-\init,0) to ++(\al,0) node[right, color=black] {Softmax};\\
                \node (cir1) at (\init/2,.1) [circle,inner sep=0.5pt, draw,minimum width=.1 cm,minimum height=.1 cm, fill=none] {\scriptsize $+$} node[right, color=black] {Elementwise addition};\\[.5 ex]
                \node (cir2) at (\init/2,.15) [circle,inner sep=0.5pt, draw,minimum width=.1 cm,minimum height=.1 cm, fill=none] {\scriptsize $\times$} node[right, color=black] {Elementwise multiplication};\\
                \node (mod1) at (0,.2) [draw,minimum width=\init/2 cm,minimum height=.2 cm, fill=none] {\footnotesize MOD1} node[right, color=black] {$3 \times 3$ avgpool, $/2$, $+1$};\\[.2 ex]
                \node (mod2) at (0,.25) [draw,minimum width=\init/2 cm,minimum height=.2 cm, fill=none] {\footnotesize MOD2} node[right, color=black] {$2 \times 2$ avgpool, $/2$};\\
                \node (mod22) at (0,.25) [draw=none,minimum width=\init/2 cm,minimum height=.2 cm, fill=none] {} node[right, color=black] {$1 \times 1$ conv, BN, ReLU};\\
            };
        \end{tikzpicture}
    \end{subfigure}
    \label{Fig:network}
\end{figure*}
\end{document}
