\begin{figure}[]
    \newcommand*{\wdth}{2.8}%
    \newcommand*{\twdth}{.19}%
    \centering
    \begin{subfigure}[]{\twdth \textwidth}
        \centering
        \includegraphics[width=\wdth cm]{./Figures/results/img0.png}
        \includegraphics[width=\wdth cm]{./Figures/results/img1.png}
        \includegraphics[width=\wdth cm]{./Figures/results/img5.png}
        \includegraphics[width=\wdth cm]{./Figures/results/img8.png}
        \includegraphics[width=\wdth cm]{./Figures/results/img13.png}
        \includegraphics[width=\wdth cm]{./Figures/results/img18.png}
        \caption{Input images}
    \end{subfigure}
    \begin{subfigure}[]{\twdth \textwidth}
        \centering
        \includegraphics[width=\wdth cm]{./Figures/results/na2014_0.pdf}
        \includegraphics[width=\wdth cm]{./Figures/results/na2014_1.pdf}
        \includegraphics[width=\wdth cm]{./Figures/results/na2014_5.pdf}
        \includegraphics[width=\wdth cm]{./Figures/results/na2014_8.pdf}
        \includegraphics[width=\wdth cm]{./Figures/results/na2014_13.pdf}
        \includegraphics[width=\wdth cm]{./Figures/results/na2014_18.pdf}
        \caption{MWA}
        \label{Fig:na2014}
    \end{subfigure}
    \begin{subfigure}[]{\twdth \textwidth}
        \centering
        \includegraphics[width=\wdth cm]{./Figures/results/mrcnn_0.png}
        \includegraphics[width=\wdth cm]{./Figures/results/mrcnn_1.png}
        \includegraphics[width=\wdth cm]{./Figures/results/mrcnn_5.png}
        \includegraphics[width=\wdth cm]{./Figures/results/mrcnn_8.png}
        \includegraphics[width=\wdth cm]{./Figures/results/mrcnn_13.png}
        \includegraphics[width=\wdth cm]{./Figures/results/mrcnn_18.png}
        \caption{MRCNN}
        \label{Fig:mrcnn}
    \end{subfigure}
    \begin{subfigure}[]{\twdth \textwidth}
        \centering
        \includegraphics[width=\wdth cm]{./Figures/results/mrcnn_w_0.png}
        \includegraphics[width=\wdth cm]{./Figures/results/mrcnn_w_1.png}
        \includegraphics[width=\wdth cm]{./Figures/results/mrcnn_w_5.png}
        \includegraphics[width=\wdth cm]{./Figures/results/mrcnn_w_8.png}
        \includegraphics[width=\wdth cm]{./Figures/results/mrcnn_w_13.png}
        \includegraphics[width=\wdth cm]{./Figures/results/mrcnn_w_18.png}
        \caption{MRCNN2}
        \label{Fig:mrcnn_w}
    \end{subfigure}
    \begin{subfigure}[]{\twdth \textwidth}
        \centering
        \includegraphics[width=\wdth cm]{./Figures/results/img0_res.pdf}
        \includegraphics[width=\wdth cm]{./Figures/results/img1_res.pdf}
        \includegraphics[width=\wdth cm]{./Figures/results/img5_res.pdf}
        \includegraphics[width=\wdth cm]{./Figures/results/img8_res.pdf}
        \includegraphics[width=\wdth cm]{./Figures/results/img13_res.pdf}
        \includegraphics[width=\wdth cm]{./Figures/results/img18_res.pdf}
        \caption{Our algorithm}
        \label{Fig:ours}
    \end{subfigure}
    \caption{ Segmentation results of teeth images using various methods~(Third image is reprinted from~\cite{T005}, the fourth image is reprinted from~\cite{T008}, and the sixth image is reprinted from~\cite{T018}). MWA in (b) is the modified watershed algorithm~\cite{Na:2014LteethMorph}. MRCNN and MRCNN2 are methods based on Mask R-CNN in~\cite{He:2018:MRCNN,Zhu:2020:teethMaskrcnn} and~\cite{Kim:2020}, respectively. In (e), the results using our algorithm are presented. }
    \label{Fig:results}
\end{figure}