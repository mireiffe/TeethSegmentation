% \begin{figure}
%     \centering
%     \begin{subfigure}{1\textwidth}
%         \centering
%         \includegraphics[width=10cm]{./Figures/Fig1_img.png}
%         \caption{A teeth image \cite{T001}}
%         \label{fig:1a}
%     \end{subfigure} 
%     \newline 
    
%     \vspace{1mm}
    
%     \begin{subfigure}{.165\textwidth}
%         \centering
%         \includegraphics[width=2.4cm]{./Figures/Fig1_img1.png}
%         \includegraphics[width=2.4cm]{./Figures/Fig1_img3.png}
%         \includegraphics[width=2.4cm]{./Figures/Fig1_img2.png}
%         \caption{}
%     \end{subfigure}
%     \begin{subfigure}{.165\textwidth}
%         \centering
%         \includegraphics[width=2.4cm]{./Figures/Fig1_sbl1.png}
%         \includegraphics[width=2.4cm]{./Figures/Fig1_sbl3.png}
%         \includegraphics[width=2.4cm]{./Figures/Fig1_sbl2.png}
%         \caption{Sobel}
%     \end{subfigure}
%     \begin{subfigure}{.165\textwidth}
%         \centering
%         \includegraphics[width=2.4cm]{./Figures/Fig1_cny1.png}
%         \includegraphics[width=2.4cm]{./Figures/Fig1_cny3.png}
%         \includegraphics[width=2.4cm]{./Figures/Fig1_cny2.png}
%         \caption{Canny}
%     \end{subfigure}
%     \begin{subfigure}{.165\textwidth}
%         \centering
%         \includegraphics[width=2.4cm]{./Figures/Fig1_er1.png}
%         \includegraphics[width=2.4cm]{./Figures/Fig1_er3.png}
%         \includegraphics[width=2.4cm]{./Figures/Fig1_er2.png}
%         \caption{$\oma$}
%         \label{Fig:1d}
%     \end{subfigure}
%     \caption{(a) A teeth image and (b) sub-images in the green boxes. (c--d) Results of Sobel and Canny edge detectors with automatically chosen thresholds by MATLAB. (e) $\oma$ from GADF. Weak edges appear at the boundary between teeth and gums, the boundary between two teeth with similar color and brightness, and the junction of boundaries. As shown in the results, GADF can detect tooth boundaries much better than the other methods while there are many unnecessary parts. }
%     \label{Fig:weak_edges}
% \end{figure}
\begin{figure}
    \centering
    \begin{subfigure}{1\textwidth}
        \centering
        \includegraphics[width=10cm]{./Figures/Fig1_img.png}
        \caption{A teeth image \cite{T001}}
        \label{fig:1a}
    \end{subfigure} 
    \newline 
    
    \vspace{1mm}
    
    \begin{subfigure}{.165\textwidth}
        \centering
        \includegraphics[width=2.4cm]{./Figures/Fig1_img1.png}
        \includegraphics[width=2.4cm]{./Figures/Fig1_img3.png}
        \includegraphics[width=2.4cm]{./Figures/Fig1_img2.png}
        \caption{}
    \end{subfigure}

    \caption{(a) A teeth image and (b) sub-images in the green boxes. (c--d) Results of Sobel and Canny edge detectors with automatically chosen thresholds by MATLAB. (e) $\oma$ from GADF. Weak edges appear at the boundary between teeth and gums, the boundary between two teeth with similar color and brightness, and the junction of boundaries. As shown in the results, GADF can detect tooth boundaries much better than the other methods while there are many unnecessary parts. }
    \label{Fig:edge_regions}
\end{figure}